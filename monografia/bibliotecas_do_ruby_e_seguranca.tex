Este capitulo tem o objetivo de mostrar alguns conceitos básicos sobre as bibliotecas do
\emph{Ruby}, e os riscos e procedimento de segurança que existem para elas.
Na seção \ref{bibliotecas_do_ruby} vamos apresentar alguns conceitos das bibliotecas do
\emph{Ruby}, e no fim do capítulo na seção \ref{segurança_ruby} vamos falar um pouco
sobre a segurança de bibliotecas do \emph{Ruby}.

\section{Bibliotecas do Ruby}
\label{bibliotecas_do_ruby}

Assim como muitas linguagens como por exemplo \emph{C}, \emph{C++}, \emph{Java}, \emph{Python} e muitas
outras, o \emph{Ruby} também possui um vasto conjunto de bibliotecas, sendo que a maior parte
delas é distribuída na forma de \emph{gem}. Seu precoesso de instalação é feito por meio do programa
\emph{gem} que será explicação na sub-seção \ref{subsection:gem}. Também existe um menor número de
bibliotecas que são lançadas como arquivos compactados em ‘‘.zip'' ou ‘‘.tar.gz''. Seu processo
de instalação geralmente é feito por meio de arquivos de ‘‘README'' ou ‘‘INSTALL'' que possuem
instruções de instalação [\citeonline{libraries_ruby}].

\subsection{O programa gem}
\label{subsection:gem}

O \emph{\href{https://rubygems.org/}{gem}} \footnote{gem: \url{https://rubygems.org/}} é um sistema
de pacotes do \emph{Ruby} desenvolvido para facilitar a criação, o compartilhamento, e a instalação de
bibliotecas. O \emph{gem} possui características sinilares ao sistema de distruição de pacotes
\emph{\href{https://packages.qa.debian.org/a/apt.html}{apt-get}}, no entanto ao invés de fazer a distribuição
de pacotes para \emph{\href{https://www.debian.org/}{Debian GNU/Linux distribution}} e seus variantes, ele faz
a distribuição de pacotes \emph{Ruby} [\citeonline{libraries_ruby}].

O projeto do \emph{RubyGems}, que desenvolve a ferramenta \emph{gem}, foi criado em abril de 2009 por
\emph{\href{https://twitter.com/qrush}{Nick Quaranto}} \footnote{Nick Quaranto: \url{https://twitter.com/qrush}}.
Com o tempo cresceu atingindo mais de 115 \emph{Rubyistis} e milhões de gemas baixadas. Até a versão
‘‘\emph{1.3.6}'' o \emph{RubyGems} possuía o nome \emph{Gemcutter}, sendo renomaeda a partir desta versão
para \emph{RubyGems}, objetivando a solidificação do papel central do site na comunidade do \emph{Ruby}
[\citeonline{about_rubygems}].

A instalação do \emph{gem} pode ser feita pelo terminar executando ‘‘\emph{sudo apt-get install gem}'' com
privilégios de administrador ou acessando \url{https://rubygems.org/pages/download/} e baixando a última
versão do \emph{RubyGems}. Para o caso de baixar a instalação, depois deve-se descompactar o pacote, entrar
no diretório e executar ‘‘\emph{ruby setup.rb}'' com privilégios de administrador.

Caso haja uma versão do \emph{gem} instalada, pode-se fazer a atualização para a última versão executando
‘‘\emph{gem update --system}'' com privilégios de administrador.

Se ocorrer algum problema no momento da instalação, atualização ou se necessitar de mais
alguma informações, se pode executar ‘‘\emph{ruby setup.rb --help}'' para obter ajuda.

A partir da versão ‘‘\emph{1.9}'' do \emph{Ruby} esse proceso de instalação não é mais necessário, pois o
\emph{gem} vem por \emph{default} instalado junto com \emph{Ruby}, mas para as versões anteriores
é necessário fazer a instalação manualmente.

Após feito a instalação, a ferramenta \emph{gem} nos auxilia a fazer a busca de gemas com o comando
‘‘ \emph{gem search ‘nome da gema' }''. Esta ferramenta, também nos permite fazer a instalação de
gemas utilizando o comando ‘‘ \emph{gem install ‘nome da gema' }'' ou o comando
‘‘ \emph{gem install ‘nome da gema'.gem }'' quando o código da gema já está na nossa máquina.

Caso haja mais interesse também se pode acessar os guias do \emph{RubyGems} em
\url{http://guides.rubygems.org/rubygems-org-api/}, onde se pode aprender como o \emph{gem} funciona e como
se pode contriuir, criar e publicar novas gemas.

\section{Segurança}
\label{segurança_ruby}

A proteção de dados sempre foi uma questão muito discutida e na comunidade do \emph{Ruby} isso não é
diferente, justamente porque a todo momento estamos sujeitos a sofrer ataques e com isso podemos ser
prejudicados, perdendo dinheiro e informações sigilosas.

Sabendo destes problemas a comunidade do \emph{Ruby} possui um esquema para corrigir os problemas de
segurança. Neste esquema as vulnerabilidade descobertas são reportadas via e-mail para
\href{mailto:security@ruby-lang.org}{\nolinkurl{security@ruby-lang.org}} que é uma lista privada com
membros que administram o \emph{Ruby}, como por exemplo \emph{Ruby commiters}. Neste esquema por medidas
de segurança, os membors da lista de segurança somente compartilham as vulnerabilidades quando elas já
estão solucionadas. Neste publicação é informado o tipo de erro, os problemas que o erro causa, e a
solução que deve ser tomada para
sua correção [\citeonline{security_ruby}].

Um exemplo é a vulnerabilidade publicada em 10/04/2014 que fala sobre um grave problema na
implementação no \emph{\href{https://www.openssl.org}{OpenSSL’s}} do \emph{TLS/DTLS} (\emph{transport
layer security protocols}) que foi referênciado com o identificador \emph{\href{https://cve.mitre.org/}{CVE}}
(\href{https://web.nvd.nist.gov/view/vuln/detail?vulnId=CVE-2014-0160}{CVE-2014-0160)}. Neste problema
uma pessoa mal intencionada pode roubar dados da memória tanto na comunicação entre o servidor com o cliente,
como na comunicação do cliente para o servidor, mas não limitado para chaves secretas usada para criptografia
\emph{SSL} e autenticação de \emph{tokens} [\citeonline{openssl_problem_security_ruby}].

\subsection{Segurança das gemas}

O foco deste trabalho não é segurança, mas vale atentar para alguns detalhes de segurança do \emph{gem} para
não ter dor de cabeça depois.

Uma \emph{gema} pode ser instalada a qualquer momento em um projeto \emph{Ruby}, deste modo o código desta
\emph{gema} será executado no contexto de uma aplicação em um servidor. Claramente isso implica em uma séria
vulnerabilidade no servidor, pois caso o autor da gema seja mal intencionado, ele pode conseguir invadir o
servidor.

\emph{RubyGems} a partir da versão ‘‘\emph{0.8.11}'' permite a assinatura criptografada de uma \emph{gema}. Essa
assinatura funciona com o comando ‘‘\emph{gem cert}'' que criar um par de chaves e empacota o dado da
assinatura dentro da \emph{gema}, e o comando ‘‘\emph{gem install}'' permite que se defina uma politica de
segurança, onde se pode verificar a chave da assinatura antes da instalação. Apesar deste método
ser benéfico, ele geralmente não é usado, pois é necessário vários passos manuais no desenvolvimento e
também não existe nenhuma medida de confiança bem definida para estas chaves de assintura
[\citeonline{guide_security_rubygems}].

Assim como o \emph{Ruby}, o \emph{RubyGems} também possui um esquema para reportar vulnerabilidades que
é composto por duas vertentes. Na primeira vertente se reporta erros na gema de outros usuários e na
segunda vertente se reporta erros da própia gema.

No caso para reportar vulnerabilidade de \emph{gemas} de outros usuários, sempre é necessário verificar
se a vulnerabilidade ainda não é conhecida. Caso ela ainda não seja conhecida, é recomendado que se
reporte o erro por um e-mail privado diretamente para o dono da gema, informando o problema e
indicando uma possível solução.

Por outro lado caso descubra uma vulnerabilidade em uma de suas \emph{gema}, primeiramente
é necessário que se requisite um identificador \emph{\href{https://cve.mitre.org/}{CVE}} via e-mail para
\href{mailto:cve-assign@mitre.org}{\nolinkurl{cve-assign@mitre.org} }, pois deste modo existirá
um identificador único para o problema. Com o identificador em mãos é necessário trabalhar em uma
possível solução. E assim que encontrar uma solução, será necessário criar um \emph{patch} de correção.
E finalmente depois de criar o \emph{patch}, deve-se informar a comunidade que existia um problema na
\emph{gema} e que essa vulnerabilidade foi corrigida no \emph{patch ‘‘x''}. Para este caso também
recomenda-se adicionar o problema em um \emph{database open source} de vulnerabilidade, como por exmeplo o
\href{http://osvdb.org/}{OSVDB}, e também enviar um e-mail para \href{mailto:ruby-talk@ruby-lang.org}
{\nolinkurl{ruby-talk@ruby-lang.org} } com o \emph{subject: ‘‘[ANN][Security]''}, informando detalhes
sobre a vulnerabilidade, as versões que possuem esse erro, e quais as ações que devem ser tomadas
para corrigir o problema.