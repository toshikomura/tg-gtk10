Este capitulo tem o objetivo de mostrar algumas definições básicas sobre as bibliotecas do
\emph{Ruby}, e os riscos e procedimento de segurança que existem para elas.
Na seção \ref{section:bibliotecas_do_ruby}, vamos apresentar alguns conceitos das bibliotecas do
\emph{Ruby}, e em seguida na seção \ref{section:segurança_do_ruby}, vamos falar um pouco
sobre a segurança do \emph{Ruby}, e por fim na seção \ref{section:segurança_das_bibliotecas_do_ruby},
vamos falar sobre a segurança das bibliotecas do \emph{Ruby}. Caso ainda não conheça a linguagem
\emph{Ruby}, pode se consultar o apêndice \ref{chapter:conceitos_e_historia_do_ruby} para se obter
algumas informações.

\section{Bibliotecas do Ruby}
\label{section:bibliotecas_do_ruby}

Assim como muitas linguagens, como por exemplo \emph{C}, \emph{C++}, \emph{Java}, \emph{Python} e muitas
outras, o \emph{Ruby} também possui um vasto conjunto de bibliotecas, sendo que a maior parte
delas é distribuída na forma de \emph{gem}. Seu precoesso de instalação é feito por meio do programa
\emph{gem}, que será explicado na sub-seção \ref{subsection:o_programa_gem} [\citeonline{libraries_ruby}].

Também existe um menor número de bibliotecas que são lançadas como arquivos compactados em ‘‘.zip''
ou ‘‘.tar.gz''. Seu processo de instalação geralmente é feito por meio de arquivos de ‘‘README''
ou ‘‘INSTALL'', que possuem as instruções de instalação [\citeonline{libraries_ruby}].

As bibliotecas do \emph{Ruby} foram apelidadas com o nome de \emph{gem} ou gemas, justamente pelo
motivo de que a maior parte das bibliotecas é desenvolvida no formato de \emph{gem}.

\subsection{O programa gem}
\label{subsection:o_programa_gem}

O \emph{\href{https://rubygems.org/}{gem}} \footnote{gem: \url{https://rubygems.org/}} é um sistema
de pacotes do \emph{Ruby}, desenvolvido para facilitar a criação, o compartilhamento, e a instalação de
bibliotecas. Esta ferramenta possui características sinilares ao sistema de distruição de pacotes
\emph{\href{https://packages.qa.debian.org/a/apt.html}{apt-get}}, no entanto ao invés de fazer a distribuição
de pacotes para \emph{\href{https://www.debian.org/}{Debian GNU/Linux distribution}} e seus variantes, ela faz
a distribuição de pacotes \emph{Ruby} [\citeonline{libraries_ruby}].

O projeto \emph{RubyGems}, desenvolvedor da ferramenta \emph{gem}, foi criado em Abril de 2009 por
\emph{\href{https://twitter.com/qrush}{Nick Quaranto}} \footnote{Nick Quaranto: \url{https://twitter.com/qrush}},
numa tentativa de proporcionar uma melhor \emph{API} para acessar as gemas [\citeonline{about_rubygems}].

Com o passar do tempo, o projeto cresceu, chegando a possuir mais de 115 \emph{Rubyistis} contribuintes e
milhões de gemas baixadas [\citeonline{about_rubygems}].

Até a versão ‘‘\emph{1.3.6}'', o nome do projeto era \emph{Gemcutter}, sendo renomaeda a partir desta versão
para \emph{RubyGems}, objetivando a solidificação do papel central do site na comunidade do \emph{Ruby}
[\citeonline{about_rubygems}].

Caso haja mais interesse, é possível consultar algumas funcionalidade da ferramenta \emph{gem} no apêndice
\ref{chapter:uso_da_ferramenta_gem}, e também se pode acessar os
\emph{\href{http://guides.rubygems.org/rubygems-org-api/}{guias do RubyGems}} 
\footnote{Guias do RubyGems: \url{http://guides.rubygems.org/rubygems-org-api/}}, para obter informações
de como o \emph{gem} funciona.

\section{Segurança do Ruby}
\label{section:segurança_do_ruby}

A proteção de dados sempre foi uma questão muito discutida e na comunidade do \emph{Ruby} isso não é
diferente, justamente porque a todo momento estamos sujeitos a sofrer ataques, e com isso podemos ser
prejudicados, perdendo dinheiro e informações sigilosas.

A comunidade do \emph{Ruby}, sabendo destes problemas, construiu um esquema para corrigir as
vulnerabilidades de segurança. Neste esquema, as vulnerabilidade descobertas são reportadas via e-mail para
\href{mailto:security@ruby-lang.org}{\nolinkurl{security@ruby-lang.org}}, que é uma lista privada com
membros que administram o \emph{Ruby}, como por exemplo \emph{Ruby commiters}. Neste esquema, os membros da
lista de segurança, por medidas de segurança, somente compartilham as vulnerabilidades quando elas já estão
solucionadas. Na publicação da vulnerabilidade é informado o tipo de erro, os problemas que o erro causa,
e a solução que deve ser tomada para sua correção [\citeonline{security_ruby}].

Um exemplo é a vulnerabilidade publicada em 10/04/2014, referênciado com o identificador
\emph{\href{https://cve.mitre.org/}{CVE}}
(\href{https://web.nvd.nist.gov/view/vuln/detail?vulnId=CVE-2014-0160}{CVE-2014-0160)}, fala sobre um grave
problema na implementação no \emph{\href{https://www.openssl.org}{OpenSSL’s}} do \emph{TLS/DTLS} (\emph{transport
layer security protocols}). Neste problema, uma pessoa mal intencionada poderia roubar dados da memória, tanto
na comunicação entre o servidor com o cliente, como na comunicação do cliente para o servidor
[\citeonline{openssl_problem_security_ruby}].

\section{Segurança das Bibliotecas do Ruby}
\label{section:segurança_das_bibliotecas_do_ruby}

Assim como na linguagem \emph{Ruby}, também é necessário tomar alguns cuidados ao se utilizar as suas
bibliotecas, pois podemos estar abrindo brechas em nossas, e com isso correndo riscos de perder
informações.

Em um primeiro momento, parece muito simples encontrar uma biblioteca do \emph{Ruby} e utilizar as
suas funcinalidades. No entanto, isso não é tão fácil, pois ao se instalar uma biblioteca,
todo o seu código é executado na máquina. Deste modo, a biblioteca pode roubar dados ou abrir
brechas com qualquer código malicioso escondido. Por este motivo, antes de instalar qualquer
biblioteca do \emph{Ruby}, devemos verificar se a fonte da gema é confiável. E isso é válido para
a instalação de qualquer biblioteca em qualquer linguagem.

O \emph{RubyGems} a partir da versão ‘‘\emph{0.8.11}'', permite a assinatura criptografada de uma \emph{gema}.
Essa assinatura funciona com o comando ‘‘\emph{gem cert}'', que criar um par de chaves e empacota o dado da
assinatura dentro da \emph{gema}. O comando ‘‘\emph{gem install}'' permite que se defina uma politica de
segurança, onde se pode verificar a chave da assinatura antes da instalação [\citeonline{guide_security_rubygems}].

Apesar deste método de chaves criptografadas ser benéfico, ele geralmente não é usado, pois é necessário
vários passos manuais no desenvolvimento, e também não existe nenhuma medida de confiança bem definida
para estas chaves de assintura [\citeonline{guide_security_rubygems}].

Assim como o \emph{Ruby}, o \emph{RubyGems} também possui um esquema para reportar vulnerabilidades.
Este esquema é divido em duas vertentes. Na primeira se reporta erros na gema de outros usuários e na
segunda se reporta erros da própia gema.

No caso para reportar vulnerabilidade de \emph{gemas} de outros usuários, sempre é necessário verificar
se a vulnerabilidade ainda não é conhecida. Caso ela ainda não seja conhecida, é recomendado que se
reporte o erro por um e-mail privado, diretamente para o dono da gema, informando o problema e
indicando uma possível solução.

Por outro lado caso descubra uma vulnerabilidade em uma de suas \emph{gema}, primeiramente
é necessário que se requisite um identificador \emph{\href{https://cve.mitre.org/}{CVE}} via e-mail para
\href{mailto:cve-assign@mitre.org}{\nolinkurl{cve-assign@mitre.org} }, pois deste modo existirá
um identificador único para o problema. Com o identificador em mãos, é necessário trabalhar em uma
possível solução. Assim que encontrar uma solução, será necessário criar um \emph{patch} de correção.
E finalmente depois de criar o \emph{patch}, deve-se informar a comunidade que existia um problema na
\emph{gema} e que essa vulnerabilidade foi corrigida no \emph{patch ‘‘x''}.

Para este segundo caso, também recomenda-se adicionar o problema em um \emph{database open source}
de vulnerabilidade, como por exmeplo o \href{http://osvdb.org/}{OSVDB}. Além disso, também recomenda-se
enviar um e-mail para \href{mailto:ruby-talk@ruby-lang.org} {\nolinkurl{ruby-talk@ruby-lang.org} } com o
\emph{subject: ‘‘[ANN][Security]''}, informando detalhes sobre a vulnerabilidade, as versões que
possuem esse erro, e as ações que devem ser tomadas para corrigir o problema.