Este capítulo tem o intuito de apresentar brevemente a história e o conceito de biblioteca. Na seção
\ref{section:historia_biblioteca} falaremos um pouco sobre a evolução do conceito de biblioteca, e
depois apresentaremos algumas definiçoes básicas sobre bibliotecas na seção \ref{section:conceitos}.
Caso tenha interesses em saber quais as classificações para bibliotecas, pode-se consultar o
apêndice \ref{chapter:classificação_de_bibliotecas}.

\section{História}
\label{section:historia_biblioteca}

Os primeiros conceitos que tinham proximidades com a definição de bibliotecas apareceram publicamente
com o \emph{software} ‘‘\emph{COMPOOL}'' (\emph{Communication Pool}) desenvolvido em \emph{JOVIAL},
que é uma linguagem de alto nível parecido com \emph{ALGOL}, mas especializada para desenvolvimento
de sistemas embarcados. O ‘‘\emph{COMPOOL}'' tinha como propósito compartilhar os dados do
sistema entre vários progrmas, fornecendo assim informação centralizada. Com essa visão o
‘‘\emph{COMPOOL}'' seguiu os princípios da ciência da computação, separando interesses e
escondendo informações [\citeonline{history_of_programming_languages}].

As linguagens \emph{COBOL} e \emph{FORTRAN} também possuiam uma prévia implementação do sistema de
bibliotecas. O \emph{COBOL} em 1959 tinha a capacidade de comportar um sistema primitivo de
bibliotecas, mas segundo \emph{Jean Sammet} esses sistema era inadequado. Já o
\emph{FORTRAN} possuia um sistema mais moderno, onde ele permitia que os subprogramas poderiam ser
compilados de forma independente um dos outros, mas com essa nova funcionalidade o compilador acabou
ficando mais fraco com relação a ligação, pois com essa possibilidade adicionada ele não conseguia
fazer a verificação de tipos entre os subprogramas [\citeonline{history_of_programming_languages}].

Por fim chegando no ano de 1965 com a linguagem \emph{Simula 67}, que foi a primeira linguagem
de programação orientada a objetos que permitia a inclusão de suas classes em arquivos de bibliotecas.
Ela também permitia que os arquivos de bibliotecas fossem utilizadas em tempo de compilação para
complementar outros programas [\citeonline{history_of_programming_languages}].

\section{Conceitos}
\label{section:conceitos}

Biblioteca do inglês \emph{library}, é um conjunto de fontes de informação que possuem recursos
semelhantes. Para a computação, uma biblioteca é um conjunto de subprogramas ou
rotinas que tem por função principal prover funcinalidades usualmente utilizadas por
desenvolvedores em um determinado contexto. Neste caso os desenvolvedores não precisam ter
nenhum conhecimento sobre o funcionamento interno das bibliotecas, mas precisam saber para que
serve cada uma destas funcionalidades e como se deve usá-las.

Geralmente as bibliotecas possuem uma \emph{API} (\emph{Application Programming Interface}), onde
é disponiblizado as suas funcionalidades e a sua forma de uso, mostrando quais são os seus
parâmetros de entrada e saída e os seus respectivos tipos.

A utilização de uma biblioteca torna-se importante, porque além de modularizar um \emph{software}, ela
permite que os desenvolvedores não se preocupem em fazer implementações repetitivas, ou seja, fazer
copias de funções de um produto para outro, e isso se deve ao fato de que quando a biblioteca for
incluída no projeto, a função já vai estar disponível para uso.