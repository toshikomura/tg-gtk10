Neste capitulo iremos apresentar alguns conceitos da linguagem \emph{Ruby} na seção \ref{ruby}, 
depois vamos falar um pouco de sua história na seção \ref{história_ruby}, também vamos ver 
a importância de uma \emph{API} na seção \ref{API}, em seguida vamos apresentar alguns conceitos 
de bibliotecas do \emph{Ruby} na seção \ref{bibliotecas_do_ruby}, e no fim vamos falar um pouco 
sobre segurança de bibliotecas do \emph{Ruby} na seção \ref{segurança_ruby}.

\section{Ruby}
\label{ruby}

\emph{\href{https://www.ruby-lang.org/en/}{Ruby}} é uma linguagem de programação 
dinâmica de código aberto com foco na simplicidade e produtividade, tem uma sintaxe elegante, é natural de 
ler e escrever, e foi inventada por \emph{ \href{http://www.rubyist.net/~matz/}{Yukihiro ‘‘Matz'' Matsumoto}}
\footnote{Yukihiro ‘‘Matz'' Matsumoto: \url{http://www.rubyist.net/~matz/}}
que tentou cuidadosamente criar uam linguagem balaceanda, misturando as suas linguagens favoritas 
( \emph{Perl}, \emph{Smalltalk}, \emph{Eiffel}, \emph{Ada}, e \emph{Lisp}) [\citeonline{ruby_site}]. 

\emph{{Yukihiro ‘‘Matz'' Matsumoto}} sempre enfativa que ele estava ‘‘tentando fazer o \emph{Ruby} natural, não simples'' e também com toda a 
sua experiência acrescenta que ‘‘\emph{Ruby} é simples na aparência, mas é muito complexo internamente, assim 
como o corpo humano.''. \emph{‘‘Matz''} também relembra que procurava uma linguagem de script que fosse mais 
poderosa que \emph{Perl} e mais orientada a objeto do que \emph{Python}, tentando encontrar a 
sintaxe ideal nas outras linguagens [\citeonline{matz_ruby_talk_main_list}] [\citeonline{matz_often_sya_of_ruby}]. 

Em muitas linguagens números e outros tipos primitivos não são \emph{objetos}, no entanto no \emph{Ruby} 
cada pedaço de informação possui propiedades e ações, ou seja, tudo é \emph{objeto}, Neste caso o 
\emph{Ruby} possui esse tipo de característica, pois ele seguiu a influência do \emph{SmallTalk}, onde 
todos seus tipos, inclusive os mais primitivos como inteiros e booleanos são objetos.

Logo abaixo no código \emph{Ruby} ‘‘\ref{lst:objeto_ruby} - Tudo é objeto para o Ruby'' segue um 
exemplo onde construimos um \emph{método} ‘‘tipo'' para o \emph{objeto} inteiro ‘‘5''. Suponde que se 
requisite o \emph{método} ‘‘\emph{5.tipo}'' o retorno seria ‘‘\emph{Tipo inteiro}''.

\lstinputlisting[ style=customRuby, caption={Tudo é objeto para o Ruby}, label={lst:objeto_ruby}]
{codigos/objeto_ruby.rb}

O \emph{Ruby} também é flexível, pois possibilita remover ou redefinir seu própio \emph{core}, por 
exemplo no código ‘‘\ref{lst:ruby_flexivel} - Ruby Flexível'' acrescentamos na \emph{classe ‘‘Numeric''} 
o \emph{método} ‘‘vezes()'' que possui a mesma característica do operador ‘‘*''. Ao se fazer a execução 
deste código a variável ‘‘y'' recebe o valor ‘‘10'' como resultado da operação ‘‘5.vezes 2''.

\lstinputlisting[ style=customRuby, caption={Ruby flexível}, label={lst:ruby_flexivel}]
{codigos/ruby_flexivel.rb}

Apesar de ser flexível esta linguagem possui algumas convenções com por exemplo no código 
‘‘\ref{lst:convencao_ruby} - Convenção Ruby'', a definição de ‘‘var'' ou qualquer outro nome de 
variável dependendo do contexto sozinha deve ser uma variável local, ‘‘@var'' deve ser uma instância 
de uma variável por causa do caracter ‘‘@'' como primeiro caracter e ‘‘\$var'' deve ser uma variável global 
por causa do caracter ‘‘\$'' como primeiro caracter.

\lstinputlisting[ style=customRuby, caption={Convenção Ruby}, label={lst:convencao_ruby}]
{codigos/convencao_ruby.rb}

\section{História}
\label{história_ruby}

O \emph{Ruby} foi criado em 24 de fevereiro de 1993 e o com o passar do tempo foi ganhando espaço na 
comunidade de desenvolvedores, chegando em 2006 a ter uma grande massa de aceitação, possuindo várias 
conferências de grupos de usuários ativos pelas principais cidades do mundo [\citeonline{ruby_site}].

O nome \emph{Ruby} veio a partir de uma conversa de chat entre \emph{Matsumoto} e \emph{Keiju Ishitsuka}, 
antes mesmo de qualquer linha de código ser escrita. Inicialmente dois nomes foram propostos ‘‘\emph{Coral}'' 
e ‘‘\emph{Ruby}''. \emph{Matsumoto} escolheu o segundo em um e-mail mais tarde. E depois de um tempo, após 
já ter escolhido o nome, ele percebeu no fato de que \emph{Ruby} era o nome da \emph{birthstone} (pedra 
preciosa que simboliza o mês de nascimento) de um de seus colegas.

Sua primeira versão foi anunciada em 21 de dezembro de 1995 na ‘‘\emph{Japanese domestic newsgroups}''. 
Subsequêncialmente em dois dias foram lançadas mais três versões juntamente com a
‘‘\emph{Japanese-Language ruby-list main-list}'' (\emph{RubyTalk}).

\emph{Ruby-Talk} \footnote{Ruby-Talk: \url{https://www.ruby-forum.com/}} a primeira \emph{lista de 
discussão da Ruby} \footnote{Listas de discussões do Ruby: 
\url{https://www.ruby-lang.org/en/community/mailing-lists/}} chegou a possuir em média 200 mensagens por dia
em 2006. E com o passar dos útlimos anos, essa média veio a cair, pois o crescimento da comunidade 
obrigou a criação de grupos especificos, empurrando os usuários da lista principal para as lista específica.


\section{API}
\label{API}

Uma \emph{Application Programming Interface (API)} possui como principal função definir de forma simplificada 
o acesso as funcinalidades de um certo componente, informando ao usuário somente para que serve cada função 
e como usá-la, indicando os parâmetros de entrada e saída com os seus respectivos tipos. 

Basicmante uma \emph{API} diminui a complexidade para o seu usuário. Por exemplo por analogia a \emph{API} 
de um elevador seria o seu painel, e a entrada seria o andar que se deseja ir. Desta forma supondo que 
um usuário deseja ir do segundo para o décimo andar, bastaria ele apertar o botão para chamar o 
elevador. Depois quando o elevador chegasse bastaria ele entrar e apertar o botão (10), e mesmo sem ter 
nenhum conhecimento mecânico do elevador, ele conseguiria obter como saída o décimo andar.

A \emph{API} do \emph{Ruby} não é diferente, ou seja, não é necessário que se tenha
conhecimento do funcionamento do \emph{core} do \emph{Ruby} 
\footnote{ Core do Ruby: \url{http://www.ruby-doc.org/core-2.1.3/}}, basta saber para que 
serve cada uma de suas funções disponibilizadas, seus parâmetro de entrada e saída com seus respectivos tipos.

\section{Bibliotecas do Ruby}
\label{bibliotecas_do_ruby}

Assim como muitas linguagens como por exemplo \emph{C}, \emph{C++}, \emph{Java}, \emph{Python} e muitas
outras, o \emph{Ruby} também possui um vasto conjunto de bibliotecas de terceiros, sendo que a maior parte
delas é distribuída na forma de \emph{gem}, e o menor número de bibliotecas são lançadas como arquivos 
compactados em ‘‘.zip'' ou ‘‘.tar.gz'' onde seu processo de instalação geralmente é feito por meio de 
arquivos de ‘‘README'' ou ‘‘INSTALL'' que possuem instruções de instalação [\citeonline{libraries_ruby}].

\subsection{O programa gem}
\label{subsection:gem}

O \emph{\href{https://rubygems.org/}{gem}} \footnote{gem: \url{https://rubygems.org/}} (RubyGems) é um sistema 
de pacotes do \emph{Ruby} desenvolvido para facilitar a criação, o compartilhamento e a instalação de 
bibliotecas. O \emph{gem} possui características sinilares ao sistema de distruição de pacotes
\emph{\href{https://packages.qa.debian.org/a/apt.html}{apt-get}}, no entanto ao invés de fazer a distribuição 
de pacotes para \emph{\href{https://www.debian.org/}{Debian GNU/Linux distribution}} e seus variantes, ele faz 
a distribuição de pacotes \emph{Ruby} [\citeonline{libraries_ruby}].

O projeto do \emph{RubyGems} foi criado em abril de 2009 por \emph{\href{https://twitter.com/qrush}{Nick Quaranto}}
\footnote{Nick Quaranto: \url{https://twitter.com/qrush}} e com o tempo cresceu atingindo mais de 115 
\emph{Rubyistis} e milhões de gemas baixadas. Até a versão ‘‘\emph{1.3.6}'' o \emph{RubyGems} possuía o nome 
\emph{Gemcutter}, sendo renomaeda a partir desta versão para \emph{RubyGems} com o objetivo de 
solidificar o papel central do site na comunidade do \emph{Ruby} [\citeonline{about_rubygems}].

A instalação do \emph{gem} pode ser feita pelo terminar executando ‘‘\emph{sudo apt-get install gem}'' com 
privilégios de administrador ou acessando \url{https://rubygems.org/pages/download/} e baixando a última 
versão do \emph{RubyGems}. Para o caso de baixar a instalação, depois deve-se descompactar o pacote, entrar 
no diretório e executar ‘‘\emph{ruby setup.rb}'' com privilégios de administrador. 

Caso haja uma versão do \emph{gem} instalada, pode-se fazer a atualização para a última versão executando 
‘‘\emph{gem update --system}'', nesse caso também é necessário possuir privilégios de administrador. 

Se ocorrer algum problema no momento de instalação, no momento de atualização ou necessitar de mais 
alguma informações, se pode executar ‘‘\emph{ruby setup.rb --help}'' para obter ajuda.

A partir da versão ‘‘\emph{1.9}'' do \emph{Ruby} esse proceso de instalação não é mais necessário, pois o 
\emph{RubyGems} vem por \emph{default} instalado junto com \emph{Ruby}, mas para as versões anteriores 
é necessário fazer a instalação manualmente.

Após feito a instalação, a ferramenta \emph{gem} nos auxilia a fazer a busca de gemas com o comando 
‘‘ \emph{gem search ‘nome da gema' }'' e também na instalação utilizando ‘‘ \emph{gem install ‘nome 
da gema' }'' ou ‘‘ \emph{gem install ‘nome da gema'.gem }'' quando o código da gema já está na nossa máquina.

Caso haja mais interesse também se pode acessar os guias do \emph{RubyGems} em 
\url{http://guides.rubygems.org/rubygems-org-api/}, onde se pode aprender como o \emph{gem} funciona e como
se pode contriuir, criar e publicar novas gemas. 

\section{Segurança}
\label{segurança_ruby}

A proteção de dados sempre foi uma questão muito discutida e na comunidade do \emph{Ruby} isso não é 
diferente, justamente porque a todo momento estamos sujeitos a sofrer ataques e com isso podemos ser 
prejudicados, perdendo dinheiro e informações sigilosas.

Sabendo destes problemas a comunidade do \emph{Ruby} possui um esquema para corrigir os problemas de 
segurança. Neste esquema as vulnerabilidade descobertas são reportadas via e-mail para
\href{mailto:security@ruby-lang.org}{\nolinkurl{security@ruby-lang.org}} que é uma lista privada com 
membros que administram o \emph{Ruby}, como por exemplo \emph{Ruby commiters}. Neste esquema por medidas 
de segurança, os membors da lista de segurança somente compartilham as vulnerabilidades quando elas já 
estão solucionadas. Neste publicação é informado o tipo de erro, os problemas que o erro causa, e a 
solução que deve ser tomada para 
sua correção [\citeonline{security_ruby}].

Um exemplo é a vulnerabilidade publicada em 10/04/2014 que fala sobre um grave problema na 
implementação no \emph{\href{https://www.openssl.org}{OpenSSL’s}} do \emph{TLS/DTLS} (\emph{transport 
layer security protocols}) que foi referênciado com o identificador \emph{\href{https://cve.mitre.org/}{CVE}} 
(\href{https://web.nvd.nist.gov/view/vuln/detail?vulnId=CVE-2014-0160}{CVE-2014-0160)}. Neste problema 
uma pessoa mal intencionada pode roubar dados da memória tanto na comunicação entre o servidor com o cliente,
como na comunicação do cliente para o servidor, mas não limitado para chaves secretas usada para criptografia
\emph{SSL} e autenticação de \emph{tokens} [\citeonline{openssl_problem_security_ruby}].

\subsection{Segurança das gemas}

O foco deste trabalho não é segurança, mas vale atentar para alguns detalhes de segurança do \emph{gem} para
não ter dor de cabeça depois.

Uma \emph{gema} pode ser instalada a qualquer momento em um projeto \emph{Ruby}, deste modo o código desta
\emph{gema} será executado no contexto de uma aplicação em um servidor. Claramente isso implica em uma séria 
vulnerabilidade no servidor, pois caso o autor da gema seja mal intencionado, ele pode conseguir invadir o 
servidor.

\emph{RubyGems} a partir da versão ‘‘\emph{0.8.11}'' permite a assinatura criptografada de uma \emph{gema}. Essa 
assinatura funciona com o comando ‘‘\emph{gem cert}'' que criar um par de chaves e empacota o dado da
assinatura dentro da \emph{gema}, e o comando ‘‘\emph{gem install}'' permite que se defina uma politica de
segurança, onde se pode verificar a chave da assinatura antes da instalação. Apesar deste método 
ser benéfico, ele geralmente não é usado, pois é necessário vários passos manuais no desenvolvimento e 
também não existe nenhuma medida de confiança bem definida para estas chaves de assintura 
[\citeonline{guide_security_rubygems}].

Assim como o \emph{Ruby}, o \emph{RubyGems} também possui um esquema para reportar vulnerabilidades que
é composto por duas vertentes. Na primeira vertente se reporta erros na gema de outros usuários e na  
segunda vertente se reporta erros da própia gema.

No caso para reportar vulnerabilidade de \emph{gemas} de outros usuários, sempre é necessário verificar 
se a vulnerabilidade ainda não é conhecida. Caso ela ainda não seja conhecida, é recomendado que se 
reporte o erro por um e-mail privado diretamente para o dono da gema, informando o problema e 
indicando uma possível solução. 

Por outro lado caso descubra uma vulnerabilidade em uma de suas \emph{gema}, primeiramente 
é necessário que se requisite um identificador \emph{\href{https://cve.mitre.org/}{CVE}} via e-mail para 
\href{mailto:cve-assign@mitre.org}{\nolinkurl{cve-assign@mitre.org} }, pois deste modo existirá
um identificador único para o problema. Com o identificador em mãos é necessário trabalhar em uma 
possível solução. E assim que encontrar uma solução, será necessário criar um \emph{patch} de correção. 
E finalmente depois de criar o \emph{patch}, deve-se informar a comunidade que existia um problema na 
\emph{gema} e que essa vulnerabilidade foi corrigida no \emph{patch ‘‘x''}. Para este caso também 
recomenda-se adicionar o problema em um \emph{database open source} de vulnerabilidade, como por exmeplo o 
\href{http://osvdb.org/}{OSVDB}, e também enviar um e-mail para \href{mailto:ruby-talk@ruby-lang.org} 
{\nolinkurl{ruby-talk@ruby-lang.org} } com o \emph{subject: ‘‘[ANN][Security]''}, informando detalhes 
sobre a vulnerabilidade, as versões que possuem esse erro, e quais as ações que devem ser tomadas 
para corrigir o problema. 