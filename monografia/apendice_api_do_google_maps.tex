Para fazer a adaptação da gema foi necessário fazer um estudo sobre como utilizar a \emph{API} do 
\emph{Google} e para isso foi utilizado como base o livro \emph{Beginning Google Maps API 3} 
[\citeonline{beginning_google_maps_api3}] e a 
\emph{\href{https://developers.google.com/maps/}{Google Maps API V3}} 
\footnote{Google Maps API V3: \url{https://developers.google.com/maps/}}, onde ambos se complementam 
ensinando os passos básicos para criar e manipular mapas do \emph{Google}. 

O \emph{Google Maps} e sua respectiva \emph{API} foram criadas por dois irmãos \emph{Lars} e 
\emph{Jens Rasmussen}, cofundadores da ‘‘\emph{Where 2 Technologies}'', compania dedicada a criação de mapas
que foi comprada pelo \emph{Google} em 2004 [\citeonline{beginning_google_maps_api3}].

Até o inicio do ano 2005 a rederização de mapas pela rede possuía um alto custo, necessitando de 
servidores altamente equipados para suportar a carga de trabalho. Mas em Fevereiro de 2005 através 
de um \emph{post} em seu \emph{blog}, o \emph{Google} anunciou uma nova solução de rederização, 
possiblitando ao usuário interagir com um mapa em uma página \emph{web}
[\citeonline{beginning_google_maps_api3}].

Depois de fazer o lançamento da nova forma para criar mapas, o \emph{Google} percebeu que 
muitos desenvolvedores gostariam de incorporar essa nova solução em seus projetos, e por esse 
movtio em Junho de 2005 anunciou a primeira versão pública da \emph{API} do \emph{Google Maps}
[\citeonline{beginning_google_maps_api3}].

O \emph{Google Maps} funciona de uma forma bem simples fazendo a criação e a manipulação do mapa
por meio de \emph{HTML}, \emph{CSS} e \emph{Javascript}. Basicamente o usuário por meio do \emph{browser}
requisita algum local do mapa informando a coordenada e o zoon desejado. Desta forma o servidor retorna a 
imagem do mapa que representa a posição requisitada [\citeonline{beginning_google_maps_api3}]. 