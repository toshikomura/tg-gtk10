Agora que sabemos como criar uma gema visto no capítulo
‘‘\ref{chapter:criacao_de_bibliotecas_do_ruby_on_rails} - Criando uma gema'', podemos partir para a
ideia de fazer modificações em uma gema que já existe, ou seja, fazer a adaptação de uma gema
adicionando novas funcionalidade que geralmente utilizamos. 

Suponha o cenário aonde utilizamos com frequência uma certa gema, no entanto apesar dela comportar várias 
funcionalidades, ela não possui tudo que desejamos. Neste caso basta fazer a adaptação desta gema, não 
sendo necessário criar uma nova gema do zero, isso desde que a funcionalidade que precisamos 
esteja no mesmo contexto abordado pela gema. 

Por exemplo se estamos utilizando uma gema matemática e ela não possuir uma função para calcular a 
raiz de um número, podemos fazer uma modificação nesta gema para que ela possa calcular a raiz. 
No entanto não faz nenhum sentido incluuir uma funcionalidade de criar mapas do \emph{Google}, pois 
não está no mesmo contexto da gema.

Para facilitar o entendimento utilizaremos para a explicação a gema 
\emph{\href{https://github.com/toshikomura/Google-Maps-for-Rails}{Google-Maps-for-Rails}} 
\footnote{Google-Maps-for-Rails : \url{https://github.com/toshikomura/Google-Maps-for-Rails}} que é um 
\emph{branch} da gema 
\emph{\href{https://github.com/apneadiving/Google-Maps-for-Rails}{Google-Maps-for-Rails}} 
\footnote{Google-Maps-for-Rails : \url{https://github.com/apneadiving/Google-Maps-for-Rails}} criada por 
\emph{\href{https://github.com/apneadiving}{Benjamin Roth}} 
\footnote{Bejamin Roth: \url{https://github.com/apneadiving}} e 
\emph{\href{https://github.com/MrRuru}{David Ruyer}} \footnote{David Ruyer: \url{https://github.com/MrRuru}}.

Essa gema criada por \emph{Bejamin Roth} e \emph{David Ruyer} tem como objetivo criar mapas de forma 
simplificada, proporcionando a inclusão de sobreposições oferecidas pelo \emph{Google} como por exemplo 
marcadores e circulos. Ela também possui um código bem flexível que permite a aceitação de outros 
provedores de mapas [\citeonline{google_maps_for_rails}].

\section{API do Google Maps} 
\label{section:api_do_google_maps} 
 
Para fazer a adaptação da gema foi necessário fazer um estudo sobre como utilizar a \emph{API} do 
\emph{Google} e para isso foi utilizado como base o livro \emph{Beginning Google Maps API 3} 
[\citeonline{beginning_google_maps_api3}] e a 
\emph{\href{https://developers.google.com/maps/}{Google Maps API V3}} 
\footnote{Google Maps API V3: \url{https://developers.google.com/maps/}}, onde ambos se complementam 
ensinando os passos básicos para criar e manipular mapas do \emph{Google}. 

O \emph{Google Maps} e sua respectiva \emph{API} foram criadas por dois irmãos \emph{Lars} e 
\emph{Jens Rasmussen}, cofundadores da ‘‘\emph{Where 2 Technologies}'', compania dedicada a criação de mapas
que foi comprada pelo \emph{Google} em 2004 [\citeonline{beginning_google_maps_api3}].

Até o inicio do ano 2005 a rederização de mapas pela rede possuía um alto custo, necessitando de 
servidores altamente equipados para suportar a carga de trabalho. Mas em Fevereiro de 2005 através 
de um \emph{post} em seu \emph{blog}, o \emph{Google} anunciou uma nova solução de rederização, 
possiblitando ao usuário interagir com um mapa em uma página \emph{web}
[\citeonline{beginning_google_maps_api3}].

Depois de fazer o lançamento da nova forma para criar mapas, o \emph{Google} percebeu que 
muitos desenvolvedores gostariam de incorporar essa nova solução em seus projetos, e por esse 
movtio em Junho de 2005 anunciou a primeira versão pública da \emph{API} do \emph{Google Maps}
[\citeonline{beginning_google_maps_api3}].

O \emph{Google Maps} funciona de uma forma bem simples fazendo a criação e a manipulação do mapa
por meio de \emph{HTML}, \emph{CSS} e \emph{Javascript}. Basicamente o usuário por meio do \emph{browser}
requisita algum local do mapa informando a coordenada e o zoon desejado. Desta forma o servidor retorna a 
imagem do mapa que representa a posição requisitada [\citeonline{beginning_google_maps_api3}]. 

\section{Engenharia Reversa}
\label{section:engenharia_reversa}

Para conseguirmos entender o funcionamento de uma gema, precisamos obrigatoriamente fazer uma tradução
do código fonte para diagramas, pois não conseguiremos fazer nenhuma modificação consistente
se não tivermos uma visão geral de seu funcionamento, e esse procedimento de tradução se chama 
\emph{engenharia reversa}.

\emph{Engenharia reversa} é um processo de análise para a extração de informações de algo que já 
existe em um modelo de abstração de alto nível. Essas informações podem estar no formato de código 
fonte ou mesmo em um executável. O processo de análise para a extração de dados deve ser feita de forma
minunciosa, pois pode ocorrer uma grande perda de recursos, caso alguma funcionalidade seja entendida de
forma incorreta. E o modelo de abstração de alto nível pode ser por exemplo um diagrama de caso de uso ou
um diagrama de sequência.

Aplicando a \emph{engenharia reversa} na gema \emph{Google-Maps-For-Rails} conseguimos obter como resultado
os diagrama de classe na imagem ‘‘Figure \ref{fig:diagrama_de_classes_google_maps_for_rails} - Diagrama de 
Classes Google-Maps-For-Rails'', o diagrama de atributos na imagem ‘‘Figure - 
\ref{fig:diagrama_de_atributos_google_maps_for_rails} - Diagrama de Atributos Google-Maps-For-Rails'' e o 
diagrama de herança na imagem ‘‘Figure - \ref{fig:diagrama_de_heranca_google_maps_for_rails} - Diagrama de 
Herança Google-Maps-For-Rails'' que serão explicados em mais detalhes logo a seguir.

Nenhum dos 3 diagramas segue os seus respectivos padrões e isso deve ao fato de que não existia espaço 
suficiente na imagem para representar o sistema da gema por completo. Por esse motivo optamos por definir 
novos diagramas que possuem características sinilares aos seus padrões, mas com algumas características
adicionais.

Para as imagens de diagrama de classes ‘‘Figure \ref{fig:diagrama_de_classes_google_maps_for_rails} - 
Diagrama de Classes Google-Maps-For-Rails'' e o diagrama de atributos na imagem ‘‘Figure - 
\ref{fig:diagrama_de_atributos_google_maps_for_rails} - Diagrama de Atributos Google-Maps-For-Rails''
as seguintes explicações são válidas:

\begin{itemize}

 \item Cada retângulo representa uma classe da gema.
 
 \item O nome ácima dos traços ‘‘-----'' representa o nome da classe.
 
 \item O símbolo ‘‘*'' do lado esquerdo do nome da classe representa que ela é \emph{superclasse} e o 
 símbolo ‘‘*'' do lado direito do nome da classe representa que ela é uma \emph{subclasse}, por exemplo
 ‘‘\emph{* Objects.BaseBuilder}'' é \emph{superclasse} de ‘‘\emph{Google.Builders.Map *}'', neste caso 
 ‘‘\emph{Google.Builders.Map *}'' é \emph{subclasse} de ‘‘\emph{* Objects.BaseBuilder}''. O mesmo 
 critério é válido para o símbolo ‘‘\$''.
 
 \item O símbolo ‘‘+'' do lado esquerdo do nome da classe representa que ela é \emph{incluída} em outra  
 \emph{classe} e o símbolo ‘‘+'' do lado direito do nome da classe representa que ela \emph{incluí} outra 
 \emph{classe}, por exemplo ‘‘\emph{+ Google.Objects.Common}'' é \emph{incluída} na \emph{classe} 
 ‘‘\emph{Google.Objects.Bound \$+}'', neste caso ‘‘\emph{Google.Objects.Bound \$+}'' \emph{incluí} a 
 \emph{classe} ‘‘\emph{+ Google.Objects.Common}''.
 
\end{itemize}

\begin{figure}[ht]
  \includegraphics[scale=0.48]{images/diagrama_de_classes_google_maps_for_rails.png}
  \caption{Diagrama de Classes Google-Maps-For-Rails}
  \label{fig:diagrama_de_classes_google_maps_for_rails}
\end{figure}

\begin{comment}
Percebe-se que o diagrama de classe na imagem ‘‘Figure \ref{fig:diagrama_de_classes_google_maps_for_rails} - 
Diagrama de Classes Google-Maps-For-Rails'' não segue os padrões de um diagrama de classes, por causa que 
no espaço disponível na imagem não seria possível inserir o sistema por completo, comprometendo o 
entendimento da gema. Por esse motivo as definições deste diagrama serão explicados logo a seguir.
\end{comment}

As seguintes explicações são válidas para o diagrama de classe na imagem ‘Figure 
\ref{fig:diagrama_de_classes_google_maps_for_rails} - Diagrama de Classes Google-Maps-For-Rails'':

\begin{itemize}

 \item O digrama não mostra os atributos das classes. 
 
 \item Todos os nomes seguidos de ‘‘(...)'' abiaxo do traço ‘‘-----'' representam os métodos da classe.
 
 \item Existem classes que não possuem os traços ‘‘-----'', neste caso elas possuem o nome delas
 seguida do símbolo ‘‘->'' e depois um nome de outra classe. Isso significa que esta classe
 possui os mesmo métodos da classe que vem depois do símbolo ‘‘->''. Por exemplo a 
 classe ‘‘\emph{Google.Builder.Circle -> Kml}'' representa a classe ‘‘\emph{Google.Builder.Circle}'' 
 e ela possuí os mesmo métodos da classe ‘‘\emph{Kml}'', ou seja, ela possuí os métodos
 ‘‘\emph{constructor()}'', ‘‘\emph{create\_...()}'' e ‘‘\emph{...\_option()}''. E também existe o caso 
 onde esta classe possuí métodos além dos da outra, e neesse caso esses métodos são colocados na linha 
 de baixo. Por exemplo a classe ‘‘\emph{Google.Builder.Polyline -> Kml}'' que é a classe 
 ‘‘\emph{Google.Builder.Polyline}'', além de possuír os métodos da classe \emph{Kml}, ela 
 possuí o método ‘‘\emph{\_build\_path()}''.
 
\end{itemize}

\begin{figure}[ht]
  \includegraphics[scale=0.48]{images/diagrama_de_atributos_google_maps_for_rails.png}
  \caption{Diagrama de Atributos Google-Maps-For-Rails}
  \label{fig:diagrama_de_atributos_google_maps_for_rails}
\end{figure}

\begin{comment}
O diagrama de atributos na imagem ‘‘Figure \ref{fig:diagrama_de_atributos_google_maps_for_rails} - 
Diagrama de Atributos Google-Maps-For-Rails'' não é um diagrama padrão de projeto, mas neste caso ele
serve como complemento do diagrama de classe ‘‘Figure \ref{fig:diagrama_de_classes_google_maps_for_rails} - 
Diagrama de Classes Google-Maps-For-Rails'', e isso também foi feito por causa do pouco
espaço disponível na imagem. Por esse motivo as definições deste diagrama serão explicados logo a seguir.
\end{comment}

As seguintes explicações são válidas para o diagrama de de atributos na imagem ‘‘Figure 
\ref{fig:diagrama_de_atributos_google_maps_for_rails} - Diagrama de Atributos Google-Maps-For-Rails'':

\begin{itemize}
 
 \item Todos os nomes abiaxo do traço ‘‘-----'' representam os atributos da classe. Também 
 existe o caso onde esse métodos são seguidos pelos símbolos ‘‘\{ ... \}'', onde o método é um 
 \emph{objeto} e os nomes separados por ‘‘,'' entre os símbolos ‘‘\{ ... \}'' são os atributos
 do \emph{objeto}.
 
 \item Existem classes que não possuem os traços ‘‘-----'', neste caso elas possuem o nome delas
 seguida do símbolo ‘‘->'' e depois um nome de outra classe. Isso significa que esta classe 
 possui os mesmo atributos da classe que vem depois do símbolo ‘‘->''. Por exemplo a 
 classe ‘‘\emph{Google.Builder.Polyline -> Kml}'' que representa a classe 
 ‘‘\emph{Google.Builder.Polyline}'', ela possuí os mesmo atributos da classe
 ‘‘\emph{Kml}'', ou seja, ela possuí os atributos 
 ‘‘\emph{args \{ url \}}'', ‘‘\emph{provider\_option}'' e ‘‘\emph{serviceObject}''. E também existe o caso 
 onde esta classe possuí atributos além dos da outra classe e neesse caso esses 
 atributos são colocados na linha de baixo. Por exemplo a classe 
 ‘‘\emph{Google.Builder.Circle -> Kml}'' que é a classe ‘‘\emph{Google.Builder.Circle}'', além de 
 possuír os atributos da classe \emph{Kml}, ela possuí o atributo 
 ‘‘\emph{args \{ lat, lng, radius \}}''.
 
\end{itemize}

\begin{figure}[ht]
  \includegraphics[scale=0.48]{images/diagrama_de_heranca_google_maps_for_rails.png}
  \caption{Diagrama de Herança Google-Maps-For-Rails}
  \label{fig:diagrama_de_heranca_google_maps_for_rails}
\end{figure}

\begin{comment}
O diagrama de herança na imagem ‘‘Figure \ref{fig:diagrama_de_heranca_google_maps_for_rails} - 
Diagrama de Herança Google-Maps-For-Rails'' não é um diagrama padrão de projeto, mas neste caso ele também
serve como complemento do diagrama de classe ‘‘Figure \ref{fig:diagrama_de_classes_google_maps_for_rails} - 
Diagrama de Classes Google-Maps-For-Rails'', e isso também foi feito por causa do pouco
espaço disponível na imagem. Por esse motivo as definições deste diagrama serão explicados logo a seguir.
\end{comment}

As seguintes explicações são válidas para o diagrama de herança na imagem ‘‘Figure 
\ref{fig:diagrama_de_heranca_google_maps_for_rails} - Diagrama de Herança Google-Maps-For-Rails'':

\begin{itemize}

 \item As linhas pretas indicam a organização da gema sendo que a classe \emph{Gmaps} é a classe principal.
 
 \item Os retângulos normais representam as classes.
 
 \item A classe \emph{Gmaps} possui os atributos \emph{Builders}, \emph{Objects} e 
 \emph{Google}, onde o \emph{Google} possui os atributos \emph{Builders} e \emph{Objects}.
 
 \item As linhas vermelhas com pontas de seta representam a herança entre duas \emph{classes}, onde 
 a classe que está com a seta, é a classe que herda as características da classe
 na outra ponta da linha.

 \item As linhas azuis com pontas de quadrado representam a inclusão de uma classe na outra, onde
 a classe que está com o quadrado, é a classe que incluí a classe que está na outra ponta da linha.
 
  \item Os retângulos que tem uma dobra no canto inferior direito representam um conjunto de classes, 
 onde estas classes possuem uma característica em comum. Por exemplo as classes ‘‘\emph{Kml}'', 
 ‘‘\emph{Polygon}'', ‘‘\emph{Polyline}'', ‘‘\emph{Circle}'', ‘‘\emph{Map}'', ‘‘\emph{Marker}'' e 
 ‘‘\emph{Bound}'' que são ‘‘\emph{Objects}'' do ‘‘\emph{Google}'', estão em um conjunto onde todas elas 
 herdam as características da classe ‘‘\emph{Base}''.
 
 \end{itemize}
 
 
\section{Entendimento da gema} 
\label{section:entendimento_da_gema} 

Agora que realizamos a \emph{engenharia reversa} da \emph{gema}, podemos analisar algumas de suas 
características que serão listadas e explicadas logo a seguir.

\begin{itemize}

 \item Apesar do ‘‘\emph{GMaps}'' ser a classe principal da gema, ela não é a mais
 importante, pois todas as funcionalidades da gema são controladas pela classe
 ‘‘\emph{Hanlder}''. A única funcionalidade da classe ‘‘\emph{GMaps}'' é fazer a chamada
 para a criação de ‘‘\emph{Handler}'', ou seja quando se requisita o método 
 ‘‘\emph{GMaps.build('Google')}'' o método verifica se o objeto ‘‘\emph{Handler}'' já
 existe, e caso ele não exista, o ‘‘\emph{GMaps}'' faz a criação chamando o método 
 ‘‘\emph{new Gmaps.Objects.Handler(type, options)}''.

 \item ‘‘\emph{Hander}'' é a classe que controla todo o funcionamento da gema e 
 basicamente ela possui dois momentos:
 
  \subitem - No primeiro momento ela prepara a estrutura da \emph{gema} para criação e manipulação
  do mapa, criando e setando os objetos de configuração, como por exemplo criando o \emph{objeto}
  ‘‘\emph{Primitives}''.
  
  \subitem - No segundo momento ela cria o mapa com as configurações e permite a manipulação do mapa, 
  possibilitando a criação e inserção de sobreposições como \emph{circles} e \emph{polylines}.
 
 \item A classe ‘‘\emph{Primitives}'' possui as definições que são comuns na gema, 
 como por exemplo, é ela possui a definição do tipo ‘‘\emph{Marker: google.maps.Marker}'' que 
 é a classe \emph{Marker} do \emph{Google Maps}.
 
 \item O atributo ‘‘\emph{serviceObject}'' de todas as classes de 
 ‘‘\emph{Builders}'' do ‘‘\emph{Google}'', representam o atributo que recebe o objeto do 
 \emph{Google Maps}.
 
\end{itemize}
 
\section{Adaptações}
\label{section:adaptações}

Agora que temos uma abstração de alto nível para a gema \emph{Google-Maps-For-Rails} podemos partir para a 
adaptação dela, ou seja, agora que temos alguns diagramas que nos auxiliam a visualizar o funcionamento 
geral da gema, podemos tentar acrescentar novas funcionalidades, analisando os locais das possíveis 
modificações e os impactos que essas mudanças podem causar. 

A gema já possuí sobreposições como \emph{markers} e \emph{circles}, mas até o momento não possuí a 
funcinalidade de criar direções entre um ponto de origem e um ponto de destino. Contudo a ideia é criar 
uma funcionalidade que receba como parâmetro um local de origem e um local de destino e retorne como
resultado uma sequência de ruas e direções a serem seguidas para ir do local de origem ao local de destino.

Para realizarmos essa modificação foi necessário consultar a \emph{API} do 
\emph{\href{https://developers.google.com/maps/documentation/javascript/directions}{Direction Service}} 
\footnote{Direction Service: \url{https://developers.google.com/maps/documentation/javascript/directions}}
(\emph{Serviço de Direção}) do \emph{Google}, onde verificamos que seria necessário o uso de pelo menos
quatro \emph{classes} que serão listadas e explicadas logo a seguir:

\begin{itemize}

 \item ‘‘\emph{DirectionService}'' (\emph{google.maps.DirectionsService}) que é a classe que tem o 
 objetivo de requisitar e receber o caminho entre o local de origem e o local de destino.
 
 \item ‘‘\emph{DirectionRender}'' (\emph{google.maps.DirectionsRenderer}) que é a classe que tem o 
 objetivo de rederizar no mapa o caminho entre o local de origem e o local de destino.
 
 \item ‘‘\emph{TravelMode}'' (\emph{google.maps.TravelMode}) que é a classe que tem o objetivo de 
 informar a forma como esse caminho deve ser percorrido, que pode ser caminhando (walking), de carro 
 (driving), bicicleta (bicycling) e/ou por meios de locomoção públicos (transit).
 
 \item ‘‘\emph{DirectionsStatus}'' (\emph{google.maps.DirectionsStatus}) que é a classe que tem o
 objetivo de informar o \emph{status} da requisição feita pela objeto da classe
 ‘‘\emph{DirectionService}''.
 
\end{itemize}

Sabendo da necessidade da inclusão de ‘‘\emph{DirectionService}'', ‘‘\emph{DirectionRender}'',
‘‘\emph{TravelMode}'' e ‘‘\emph{DirectionsStatus}'', decidimos que a primeira modificação na gema seria 
incluir estas quatros classe nas definições da classe ‘‘\emph{Primitives}''. E isso foi 
feita da seguinte forma como apresentado no código ‘‘Código 
\ref{lst:classe_primitives_com_atributos_de_directions} - Classe Primitives com atributo de Directions'' 
logo abaixo.

\lstinputlisting[ style=customCoffee, caption={Classe Primitives com atributo de Directions}, label={lst:classe_primitives_com_atributos_de_directions}]
{codigos/classe_primitives_com_atributos_de_directions.coffee}

\begin{figure}[ht]
  \includegraphics[scale=0.48]{images/novo_diagrama_de_heranca_google_maps_for_rails.png}
  \caption{Novo Diagrama de Herança Google-Maps-For-Rails}
  \label{fig:novo_diagrama_de_heranca_google_maps_for_rails}
\end{figure}

Em seguida criamos quatro \emph{classes} para o ‘‘\emph{Google}'', sendo que duas são ‘‘\emph{Builders}'' de 
‘‘\emph{DirectionService}'' e ‘‘\emph{DirectionRender}'', e as outras duas são ‘‘\emph{Objects}'' também das 
classes ‘‘\emph{DirectionService}'' e ‘‘\emph{DirectionRender}''. Para facilitar a compreensão, elaboramos o 
diagrama representado na imagem ‘‘Figure \ref{fig:novo_diagrama_de_heranca_google_maps_for_rails} - Novo 
Diagrama de HrançaGoogle-Maps-For-Rails'' para mostrar o local aonde inserimos as classes e quais as 
dependências que elas possuem. No caso este diagrama é o mesmo diagrama de herança que desenvolvemos na 
\emph{engenharia reversa}, mostrado na imagem ‘‘Figure \ref{fig:diagrama_de_heranca_google_maps_for_rails} - 
Diagrama de Herança Google-Maps-For-Rails'', com a adição das quatro classes que são representadas 
por retângulos tracejados.

Agora que acrescentamos estas quatro classes na gema adaptada, devemos adicionar no ‘‘\emph{Handler}'', 
novas funções para maminpular essas classes. E neste caso inserimos as funçãoes ‘‘\emph{addDirection()}'' e 
‘‘\emph{calculate\_route()}'' que podem ser vistas no código ‘‘Código 
\ref{lst:funcoes_adicionais_do_handler} - Funções adicionais do Handler'' que será explicado logo a seguir.

\begin{itemize}

 \item Na linha ‘‘2'' é adicionado a função ‘‘\emph{addDirection}'' que recebe como parâmetro 
 ‘‘\emph{direction\_data}'', que possui o informações do local de origem e local de destino que são 
 obrigatórios para criar a direção, e ‘‘\emph{provider\_options}'' que pode conter as opções da forma
 como esse direção deve ser gerada. Essa função tem por objetivo criar as direções e colocá-las no mapas.
 
 \item Na linha ‘‘13'' é adicionado a função ‘‘\emph{calculate\_route}'' com o parâmetro 
 ‘‘\emph{direction\_data}'' que tem por principal objtivo fazer a requisição para o \emph{Google } de uma 
 possível direção entre o local de origem ao local de destino.
 
 \item Da linha ‘‘3'' a linha ‘‘10'' é o conteúdo da função ‘‘\emph{addDirection}'', onde se cria os
 atributos ‘‘\emph{direction\_service}'' e ‘‘\emph{direction\_render}'' para o ‘‘\emph{Handler}'', e
 para cada um deles é atribuido o seu respectivo objeto do \emph{Google Maps} com a chamada da função 
 ‘‘\emph{@\_builder(...)}''. Depois é feito a chamada da função ‘‘\emph{calulate\_route(...)}''
 para encontrar uma possível direção e finalmente com o código 
 ‘‘\emph{@direction\_render.getServiceObject().setMap(@getMap())}'' a direção encontrada é colocado 
 no mapa.
 
 \item Da linha ‘‘13'' a linha ‘‘20'' é o conteúdo da função ‘‘\emph{calculate\_route(...)}'', onde 
 inicialmente é colocado em uma variável local o \emph{status Ok} de requisição que será utilizado
 para verificar se a requisição de direção foi feita com sucesso, e também é criada uma variável local 
 para o \emph{objeto DirectionRender} do \emph{Google Maps}. Em seguida através da chamada da função 
 ‘‘\emph{route(...)}'' do \emph{objeto DirectionService} que passa como parâmetro o local de origem e
 o local de destino e recebe como resposta o \emph{status} da requisição e o \emph{response} que pode
 conter o caminho solução. Após a execução desta função é feita a verificação da resposta, e caso 
 ela venha como o \emph{status Ok} o \emph{objeto DirectionRender} recebe o caminho solução.
 
\end{itemize}

\lstinputlisting[ style=customCoffee, caption={Funções adicionais do Handler}, label={lst:funcoes_adicionais_do_handler}]
{codigos/funcoes_adicionais_do_handler.coffee}

\section{Exemplo de uso de Google-Maps-for-Rails}
\label{section:exemplo_de_uso_de_google-maps-for-rails}

Como exemplo de uso da gema ‘‘\emph{Google-Maps-for-Rails}'' adaptada criamos o projeto 
‘‘\emph{\href{https://github.com/toshikomura/DiseasesMap}{DiseasesMap}}'' 
\footnote{DiseasesMap : \url{https://github.com/toshikomura/DiseasesMap}} que tem como objetivo representar 
a frequência de doenças no mapa do \emph{Google} utilizando sobreposições. Até o momento de término deste
trabalho essa função ainda não havia sido implementada, mas fizemos o uso da funcionalidade de direções.

Inicialmente fizemos a instalção e inclusão da gema adaptada no arquivo \emph{Gemfile} do projeto. Depois 
criamos uma estrutura básica de \emph{model/view/controller} de ‘‘\emph{locations}''. 

Em seguida para fazer o uso da função de direções utilizamos o código‘‘Código 
\ref{lst:exemplo_coffeescript_que_cria_mapa_com_direcao} - Exemplo CoffeeScript que Cria Mapa com 
Direção'' explicado logo abaixo.

\lstinputlisting[ style=customCoffee, caption={Exemplo CoffeeScript que Cria Mapa com Direção}, label={lst:exemplo_coffeescript_que_cria_mapa_com_direcao}]
{codigos/DiseasesMap/app/assets/javascripts/locations.js.coffee}

\begin{itemize}

 \item Na linha ‘‘6'' é feita a preparação da estrtura de configuração do mapa com a chamada 
 ‘‘\emph{GMaps.build('Google')}'', sendo feita a criação do \emph{objeto Handler}, que é atribuida a 
 variável local ‘‘\emph{handler}'', juntamente com as outras configurações básicas que o mapa necessita.
 
 \item Na linha ‘‘7'' é feita a chamada de ‘‘\emph{handler.buildMap(...)}'' que tem como função fazer a 
 criação do mapa a parir das configurações básicas já definidas quando foi feita a chamada de 
 ‘‘\emph{GMaps.build('Google')}''. São passados como parâmetros as variáveis, ‘‘\emph{provider}''
 que neste caso está vazio, e internal que define o ‘‘\emph{id}'' do mapa como ‘‘\emph{map}'', neste caso
 o ‘‘\emph{id}'' serve para identificar o mapa a ser modificado.
 
 \item Na linha ‘‘11'' é criado uma \emph{function} determinada pelo símbolo ‘‘->'', onde ela somente será 
 executada depois que a função ‘‘\emph{handler.buildMap(...)}'' terminar, ou seja, quando toda a criação do mapa for 
 concluída. Neste caso essa função executa as seguintes operações:
 
  \subitem Na linha ‘‘12'' é feita a criação de um \emph{marker} com a chamada da função 
  ‘‘\emph{handler.addMarker(...)}'', sendo passado como parâmetro, a sua posição que no caso é (0,0) 
  definido ‘‘\emph{lat}'' e ‘‘\emph{lng}'', a sua imagem definida por ‘‘\emph{picture}'', e sua informação 
  definido por ‘‘\emph{infowindow}''.
 
  \subitem Na linha ‘‘22'' é feita a extensão de fronteiras incluindo o novo \emph{marker} com a chamada 
  da função ‘‘\emph{handler.bounds.extendWith(...)}'', sendo passado como parâmetro o \emph{marker} criado
  anteriormente.
 
  \subitem Na linha ‘‘23'' é feita a criação de direções com a chamada da função 
  ‘‘\emph{handler.addDirection(...)}''que incluimos no ‘‘\emph{Handler}'', sendo passado como parâmetro, um 
  local de origem definido por ‘‘\emph{ origin: ‘‘São Paulo''} '', e um local de destino definido por 
  ‘‘\emph{destination: ‘‘Curitiba''} ''.
 
\end{itemize}

E para mostrar o mapa na view de ‘‘\emph{locations}'' adicionamos o código mostrado em ‘‘Código 
\ref{lst:exemplo_locations_view_que_cria_mapa_com_direcao} - Exemplo Locations view que Cria Mapa com Direção'' 
que é parte do código da \emph{view}, explicado logo a seguir.

\lstinputlisting[ style=customRubyHTML, caption={Exemplo Locations view que Cria Mapa com Direção}, label={lst:exemplo_locations_view_que_cria_mapa_com_direcao}]
{codigos/index_simplificado.html.erb}

\begin{itemize}

 \item Na linha ‘‘1'' com a \emph{tag} \emph{<h1>...</h1>} é definido como título principal da \emph{view} o
 texto ‘‘\emph{Listining locations}''.
 
 \item Os ‘‘\emph{...}'' indica que existe código, mas por simplificação na explicação, ele não fo mostrado.
 
 \item Da linha ‘‘3'' a ‘‘5'' é definido uma \emph{div} com ‘‘800px'' de largura, e dentro dessa \emph{div}
 é definido o local para a criação do mapa com ‘‘800px'' de largura e ‘‘400px'' de altura. No caso o local 
 de criação do mapa é referênciado pelo atributo \emph{id} que o mesmo \emph{id} utilizado no código 
 ‘‘Código \ref{lst:exemplo_coffeescript_que_cria_mapa_com_direcao} - Exemplo CoffeeScript que Cria Mapa com 
 Direção'' na linha ‘‘10''.  
 
\end{itemize}

Como resultado ao se acessar o \emph{index} de locations obtemos como resultado a imagem ‘‘Figure 
\ref{fig:caminho_entre_sao_paulo_e_curitiba} - Caminho entre São Paulo e Curitiba''. Neste caso a
nossa gema adaptada com a nova funcionalidade de direções funcionou corretamente, pois o caminho
mostrado é entre ‘‘São Paulo'' e ‘‘Curitiba'' como requisitamos na linha ‘‘24'' do código ‘‘Código 
\ref{lst:exemplo_coffeescript_que_cria_mapa_com_direcao} - Exemplo CoffeeScript que Cria Mapa com Direção''.

 \begin{figure}[ht]
  \includegraphics[scale=0.44]{images/caminho_entre_sao_paulo_e_curitiba.png}
  \caption{Caminho entre São Paulo e Curitiba}
  \label{fig:caminho_entre_sao_paulo_e_curitiba}
\end{figure}