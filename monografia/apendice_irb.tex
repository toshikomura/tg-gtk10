O \emph{\href{http://www.ruby-doc.org/stdlib-2.0/libdoc/irb/rdoc/IRB.html}{IRB}} 
\footnote{IRB: \url{http://www.ruby-doc.org/stdlib-2.0/libdoc/irb/rdoc/IRB.html}}
(\emph{Interactive Ruby Shell}) é uma ferramenta do \emph{Ruby} que serve para executar expressões
interativamente, fazendo a leitura da entrada padrão [\citeonline{irb_doc}].

Caso se deseje fazer os testes de uma gema manualmente sem a necessidade de inclui-lá em um projeto, podemos
fazer o uso do ‘‘\emph{IRB}'', chamando o comando ‘‘\emph{irb}''.
% o seguinte comando mostrado no código ‘‘Código \ref{lst:executa_irb} - Executa IRB'' no terminal para iniciá-lo.

\begin{comment}
\lstinputlisting[ style=customBash, caption={Executa IRB}, label={lst:executa_irb}]
{codigos/executa_irb.sh}
\end{comment}

Um exemplo de uso do \emph{IRB} pode ser visto 
% na imagem ‘‘Figure \ref{fig:exemplo_de_uso_do_irb} - Exemplo de Uso do IRB'' 
no código \ref{lst:exemplo_de_uso_do_irb}
abaixo, explicado em mais detalhes na listagem abaixo. 

\begin{comment}
\begin{figure}[ht]
  \includegraphics[scale=0.3]{images/exemplo_de_uso_do_irb}
  \caption{Exemplo de Uso do IRB}
  \label{fig:exemplo_de_uso_do_irb}
\end{figure}
\end{comment}

\lstinputlisting[ style=customBash, caption={Exemplo de uso do IRB}, label={lst:exemplo_de_uso_do_irb}]
{codigos/exemplo_de_uso_do_irb.sh}

\begin{itemize}

\begin{comment}
 \item Primeiramente é feita a chamada da ferramenta \emph{IRB} com o comando ‘‘\emph{irb}'' no terminal.
 
 \item Depois é requisitado a soma entre ‘‘\emph{1 + 1}'' resultando em ‘‘\emph{2}''.
 
 \item Em seguida é perguntado se ‘‘\emph{1 == 1}'' resultando em ‘‘\emph{true}''.

 \item E no fim é criado uma função chamada de ‘‘\emph{hello}'' com o parâmetro ‘‘\emph{name}'' e ao se 
 chamar essa função é devolvido na tela ‘‘\emph{Hello +}'' o parâmetro passado para a função. O resultado
 pode ser visto quando se requisita ‘‘\emph{hello(‘‘Maria'')}'' e se obtem como resultado 
 ‘‘\emph{Hello Maria}''.
\end{comment}
 
  \item Primeiramente na linha ‘‘1'' é feita a chamada da ferramenta \emph{IRB} com o comando ‘‘\emph{irb}'' 
  no terminal.
 
 \item Depois na linha ‘‘2'' é requisitado a soma entre ‘‘\emph{1 + 1}'' resultando em ‘‘\emph{2}'' na linha 
 ‘‘3''.
 
 \item Em seguida na linha ‘‘4'' é verificado se ‘‘\emph{1 == 1}'' resultando em ‘‘\emph{true}'' na linha 
 ‘‘5''.

 \item E no fim entre as linhas ‘‘6'' e ‘‘8'' é criado uma função chamada de ‘‘\emph{hello}'' com o 
 parâmetro ‘‘\emph{name}'' e ao se chamar essa função é devolvido na tela ‘‘\emph{Hello}'' mais o 
 parâmetro passado para a função. O resultado pode ser visto quando se requisita 
 ‘‘\emph{hello(‘‘Maria'')}'' na linha ‘‘10'', obtendo como resultado ‘‘\emph{Hello Maria}'' na linha ‘‘11''.
 
\end{itemize}

No nosso exemplo da gema ‘‘\emph{gemtranslatetoenglish}'' fizemos alguns testes simples mostrados
na código \ref{lst:teste_irb_da_gema_gemtranslatetoenglish} explicado com mais detalhes nos itens abaixo.

\begin{comment}
\begin{figure}[ht]
  \includegraphics[scale=0.28]{images/teste_irb_da_gema_gemtranslatetoenglish.png}
  \caption{Teste IRB da gema gemtranslatetoenglish}
  \label{fig:teste_irb_da_gema_gemtranslatetoenglish}
\end{figure}
\end{comment}

\lstinputlisting[ style=customBash, caption={Teste IRB da gema gemtranslatetoenglish}, label={lst:teste_irb_da_gema_gemtranslatetoenglish}]
{codigos/teste_irb_da_gema_gemtranslatetoenglish.sh}

\begin{itemize}

 \item Primeiramente na linha ‘‘1'' é feita a chamada da ferramenta \emph{IRB} com o comando ‘‘\emph{irb}'' 
  no terminal.
  
  \item Na linha ‘‘2'' é executado o comando ‘‘ \emph{require 'action\_controller'} '' para buscar a gema 
  ‘‘\emph{ActionController}'' necessária no uso da nossa gema de exemplo quando evitamos digitar a 
  \emph{PATH} completa na ‘‘\emph{view}''.

  \item Na linha ‘‘4'' é executado o comando ‘‘ \emph{require 'gemtranslatetoenglish'} '' para buscar a 
  nossa gema de exemplo.
  
  \item Na linha ‘‘6'' é executado o comando 
  ‘‘ \emph{instance\_method\_names}'' para verificar se o nosso método \emph{translate()} existe.
  
  \item Na linha ‘‘8'' é executado o comando ‘‘\emph{include}'' para incluir as funções do módulo 
  ‘‘\emph{Translatetoenglish}''.
  
  \item Na linha ‘‘10'' é executado o comando 
  ‘‘\emph{translate('Oi')}'' para verificar se a função funciona como o esperado.
  
  \item E no fim na linha ‘‘11'' podemos verificar que a função \emph{translate()} funcionou corretamente,
  pois obtemos como resultado a palavra ‘‘\emph{HELLO }''.
  
 \end{itemize}