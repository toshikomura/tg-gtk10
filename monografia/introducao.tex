Até o ano de 2005, a renderização de mapas em uma página web custava caro. Isso ocorria,
porque para apresentar um mapa, era necessário fazer a transferência de uma grande
quantidade de dados do servidor para o cliente. Isso ocorria, porque era feito a
transferência de um mapa inteiro. Também como consquência da grande quantidade de dados
transferidos, existia um alto custo no cliente para renderizar a figura.

Em Fevereiro de 2005, essa dor de cabeça foi aliviada com a criação
de uma nova solução de renderização, implementada na ferramenta \emph{Google Maps}. Esta nova
solução, consistia na divisão do mapa em pedaços. Estes pedaços do mapa, poderiam ser
requisitados um a um e colocados lado a lado para formar uma parte ou um mapa completo
]\citeonline{google_maps_power_tools_for_maximizing_the_api}]. Com essa nova renderização,
foi possível solucionar o problema da grande quantidade de dados transferidos, pois
somente se transfere as partes requisitadas do mapa, e também se solucionou o problema
da renderização, pois a quantidade de dados para se renderizar, foi reduzida razoavelmente.

Muitas companhias, percebendo o grande potêncial da ferramenta, adotaram soluções similares
que também reduziam a carga de trabalho nos servidores. Entre estas ferramentas estão, o
\emph{Yahoo Maps} do \emph{Yahoo}, o \emph{Bing Maps} da \emph{Microsoft}, e o
\emph{Yandex Maps} do \emph{Yandex}.

Para os desenvolvedores, esta nova solução de renderização, tornou-se interessante
quando se analisou as operações que poderiam ser realizadas sobre os mapas. Por exemplo, nos
mapas do \emph{Google Maps}, podemos criar um mapa em uma página web, utilizar marcadores para
marcar locais ou regiões do mapa que consideramos importante, e até determinar caminhos
para ir de um local ao outro.

Muitas empresas decidiram se mobilizar na procura de uma forma de incorporar o
\emph{Google Maps} em suas aplicações, pois perceberam que a ferramenta possuía funcionalidade
muito úteis para o dia-a-dia. O \emph{Google}, percebendo essa mobilização, resolveu facilitar
o acesso a ferramenta, divulgando em Junho de 2005 uma \emph{API}
(\emph{Application Programming Interface}) para o \emph{Google Maps}, apresentando
detalhes de como importar e utilizar as funções da ferramenta. Exemplos incluem a criação de
mapas e marcadores.

A \emph{API} foi desenvolvida para ser utilizada na linguagem \emph{Javascript}, porém rapidamente
o mercado desenvolveu \emph{APIs} de nível mais alto, que usavam a API do google para executar
funções. Por exemplo, é possível utilizar uma única função que desenha e cria uma fronteira
de visualização para marcadores, fazendo o uso das funções do \emph{Google Maps}
de criar marcadores e delimitar fronteiras.

\emph{Frameworks} de desenvolvimento também criaram \emph{APIs} próprias, cujo o objetivo é mapear
e simplificar o acesso a \emph{API} do \emph{Google Maps}. Estão entre as linguagens que possuem
estes \emph{frameworks}, o \emph{Java}, o \emph{Python}, o \emph{PHP}, o \emph{Objective-C},
o \emph{.NET}, o \emph{Ruby}, entre muitas outras.

O \emph{Ruby} possui bibliotecas que contém funcionalidade prontas para uso. Algumas destas
bibliotecas possuem funcionalidades que simplificam o acesso da \emph{API} do \emph{Google Maps}.
Basicamente estas bibliotecas mapeiam as funções da ferramenta do \emph{Google}, preparando
internamente os objetos dos mapas. Deste modo, ao utilizar uma funcionalidade da biblioteca, não
é necessário saber quais os objetos do \emph{Google Maps} precisam ser criados, pois estes
objetos serão gerados automaticamente na chamada da função.

O \emph{Ruby On Rails} é um \emph{framework} da linguagem \emph{Ruby} que facilita o
desenvolvimento de projetos \emph{web}, fazendo a criação de estruturas básicas por meio da
execução de linhas de comando, e possibilitando a importação de bibliotecas do \emph{Ruby}.

O objetivo deste trabalho, é mostrar uma forma de adaptar uma biblioteca do \emph{Ruby}
que faz o mapeamento da \emph{API} do \emph{Google Maps} e apresentar um exemplo de uso desta
biblioteca em um projeto do \emph{Ruby On Rails}. Para aprimorar o conhecimento, antes será
necessário apresentar conceitos sobre bibliotecas do \emph{Ruby}, bem como o seu modelo de
criação.

O texto apresentará um passo-a-passo da implementação de uma biblioteca simples
e funcional do \emph{Ruby} através de duas bibliotecas simples. A primeira que faz a
tradução de palavras do português para o inglês, será utilizada para mostrar os
conceitos da criação de bibliotecas no \emph{Ruby}. A segunda que faz a adição da
funcionalidade de desenhar caminhos entre dois locais utilizando uma biblioteca do
\emph{Ruby} que mapeia a \emph{API} do \emph{Google Maps}, será utilizada para apresentar
conceitos da adaptação de bibliotecas no \emph{Ruby}.

Este trabalho está organizado da seguinte maneira: bibliotecas do \emph{Ruby} no
capítulo \ref{chapter:bibliotecas_do_ruby}, onde será apresentado conceitos sobre
bibliotecas do \emph{Ruby}, criação de bibliotecas do \emph{Ruby} no capítulo
\ref{chapter:criacao_de_bibliotecas_do_ruby}, onde será apresentado um passo-a-passo
de como criar uma biblioteca do \emph{Ruby}, adaptação de bibliotecas do \emph{Ruby}
no capítulo  \ref{chapter:adaptacao_de_bibliotecas_do_ruby}, onde será apresentado um
tutorial de como adaptar uma biblioteca do \emph{Ruby}, e a conclusão no capítulo
\ref{chapter:conclusao}.
