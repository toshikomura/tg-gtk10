Até o ano de 2005 para se rederizar um mapa em uma página web, era necessário possuir servidores
altamente equipados para suportar a carga de trabalho que a tarefa exigia. No entanto, essa dor
de cabeça foi aliviada em um anúncio do \emph{Google}, com a criação da ferramenta
\emph{Google Maps} em Fevereiro de 2005. Esta ferramenta, possuía uma nova solução que reduzia
o trabalho dos servidores, possibilitando assim, a economia na compra de equipamentos.

Além da ferramenta \emph{Google Maps} do \emph{Google}, exitem outras ferramentas que possuem
soluções sinilares, como por exemplo, o \emph{Yahoo Maps} do \emph{Yahoo}, o \emph{Bing Maps}
da \emph{Microsoft} e o \emph{Yandex Maps} do \emph{Yandex}.

A rederização de mapas, torna-se interessante quando analisamos as operações que podemos
realizar sobre eles. Por exemplo nos mapas do \emph{Google Maps}. podemos criar um mapa em
uma página web, utilizar marcadores para marcar locais ou regiões do mapa que consideramos
importante. Também como exemplo, podemos determinar caminhos no mapa para ir de um local
de origem a um local de destino.

Como a solução de rederização de mapas do \emph{Google} era inovadora, muitos
desenvolvedores procuraram na época, uma forma de incorporar esta nova ferramenta em seus
projetos. E por este motivo, o \emph{Google} divulgou em Junho de 2005 uma \emph{API}
(\emph{Application Programming Interface}) para o \emph{Google Maps}.

Esta \emph{API} do \emph{Google Maps}, possui informações de como importar a biblioteca
e fazer uso de suas funções, como por exemplo, criar mapas nas páginas web, criar
marcadores, e criar caminhos entre um local de origem e um local de destino.

O \emph{Ruby On Rails} que é um \emph{framework} da linguagem \emph{Ruby}, possui
bibliotecas que simplificam o acesso da \emph{API} do \emph{Google Maps}, ou seja,
as bibliotecas do \emph{Ruby On Rails} mapeiam a \emph{API} do \emph{Google Maps}.
Esse mapeamento da \emph{API} facilita o trabalho do desenvolvedor, porque ao invés do
programador precisar criar objeto por objeto, a biblioteca do \emph{Ruby On Rails}
já cria os objetos internamente.

O objetivo deste trabalho, é mostrar como se pode criar uma biblioteca simples do 
\emph{Ruby On Rails}, e também como se pode modificar uma biblioteca
do \emph{Ruby On Rails} que trabalha com o mapeamento da \emph{API} do
\emph{Google Maps}.

Deste modo, este trabalho esta organizado da seguinte maneira: história e conceitos de
bibliotecas no capítulo \ref{chapter:historia_e_concetios_de_bibliotecas}, onde será
revisado a história das bibliotecas e explicado alguns conceitos básicos sobre elas,
conceitos de \emph{Ruby} e suas bibliotecas no capítulo \ref{chapter:ruby_e_suas_bibliotecas},
onde será apresentado alguns conceitos da linguagem \emph{Ruby}, do \emph{framework}
\emph{Ruby On Rails} e da manipulação de suas bibliotecas, criação e adaptação
de bibliotecas do \emph{Ruby On Rails} no capítulo 
\ref{chapter:criacao_e_adaptacao_de_bibliotecas_do_ruby_on_rails}, onde será apresentado um tutorial
de como criar e/ou adaptar uma biblioteca do \emph{Ruby On Rails}, e a conclusão no
capítulo \ref{chapter:conclusao}.