Até o ano de 2005, a rederização de mapas em uma página web custava caro, pois era necessário
servidores altamente equipados para suportar a carga de trabalho que a tarefa exigia.
Graças ao \emph{Google}, em Fevereiro de 2005, essa dor de cabeça foi aliviada com a criação
de uma nova solução de rederização implementada na ferramenta \emph{Google Maps}. Esta nova
solução reduzia o trabalho dos servidores, possibilitando assim, a economia de dinheiro na
compra de equipamentos.

Muitas companias, percebendo a solução revolucionária adotada pelo \emph{Google} na
ferramenta \emph{Google Maps}, não ficaram atrás, adotando soluções sinilares que
também reduziam a carga nos servidores. Estão entre estas ferramentas, o
\emph{Yahoo Maps} do \emph{Yahoo}, o \emph{Bing Maps} da \emph{Microsoft}, e o
\emph{Yandex Maps} do \emph{Yandex}.

Poara os desenvolvedores, esta nova solução de rederização, torna-se interessante
quando se analisa as operações que podem ser realizadas sobre os mapas. Por exemplo, nos mapas
do \emph{Google Maps}, podemos criar um mapa em uma página web, utilizar marcadores para
marcar locais ou regiões do mapa que consideramos importante, e até determinar caminhos
para ir de um local ao outro.

Muito desenvolvedres perceberam que a nova ferramenta do \emph{Google} poderia ser muito útil
em seus projeto, e por este motivo eles se mobilizaram na procura de uma forma de incorporar o
\emph{Google Maps} em suas aplicações. Percebendo essa mobilização, o \emph{Google} facilitou
o acesso a ferramenta, divulgando em Junho de 2005 uma \emph{API}
(\emph{Application Programming Interface}) para o \emph{Google Maps}, descrevendo como importar
e utilzar as funções da ferramenta.

O \emph{Ruby On Rails} que é um \emph{framework} da linguagem \emph{Ruby}, possui
bibliotecas que simplificam o acesso da \emph{API} do \emph{Google Maps}. Basicamente estas
bibliotecas preparam internamente os objetos do mapa e depois mapeiam as funções da
ferramenta do \emph{Google}. Por exemplo, no momento de criação do mapa, o desenvolvedor
não precisa saber quais os objetos do \emph{Google Maps} que são necessários. Ele
precisa saber somente qual a função da biblioteca que faz a criação do mapa, pois
internamente a biblioteca prepara o objeto do mapa e faz a chamada da função do
\emph{Google Maps} que cria mapas.

O objetivo deste trabalho, é mostrar alguns conceitos básicos sobre bibliotecas, bem
como a criação do mesmo, apresentando um tutorial de como se pode criar e modificar
uma biblioteca do \emph{Ruby On Rails}.

O tutorial de como se pode criar um biblioteca do \emph{Ruby On Rails}, apresentará
um passo-a-paaso da implementação de uma biblioteca simples que faz a tradução de
palavras do português para o inglês.

O tutorial de como se pode modificar uma biblioteca do \emph{Ruby On Rail}, apresentará
os passos que devem ser realizados para se adaptar uma biblioteca, utilizando como exemplo,
a adição da função de determinar caminhos entre dois locais, em uma biblioteca que mapeia a
\emph{API} do \emph{Google Maps}.

Este trabalho esta organizado da seguinte maneira: história e conceitos de
bibliotecas no capítulo \ref{chapter:historia_e_concetios_de_bibliotecas}, onde será
revisado a história das bibliotecas e explicado alguns conceitos básicos sobre elas,
bibliotecas do \emph{ruby} e segurança no capítulo \ref{chapter:bibliotecas_do_ruby_e_segurança},
onde será apresentado alguns conceitos das bibliotecas do \emph{Ruby}, criação de
bibliotecas do \emph{Ruby On Rails} no capítulo 
\ref{chapter:criacao_de_bibliotecas_do_ruby_on_rails}, onde será apresentado
um tutorial de como criar uma biblioteca do \emph{Ruby On Rails}, adaptação de
bibliotecas do \emph{Ruby On Rails} no capítulo 
\ref{chapter:adaptacao_de_bibliotecas_do_ruby_on_rails}, onde será apresentado
um tutorial de como adaptar uma biblioteca do \emph{Ruby On Rails}, e a
conclusão no capítulo \ref{chapter:conclusao}.