Nos dias de hoje com o avanço da tecnologia para satisfazer um cliente é necessário entregar um 
produto com qualidade e no menor tempo possível. Pensando na área da computação mais 
especificamente no desenvolvimento de software, existe uma grande chance de se obter a qualidade 
do produto quando se têm em mãos um projeto bem elaborado. Para se ter esse projeto, primeiramente  
se define um processo de desenvolvimento com regras a serem seguida e com o passar do tempo é 
necessário adaptar este processo para uma forma que possiblite atender as necessidades do cliente 
de uma forma mais rápida e concisa.

Pensando pelo lado da agilidade, além de aplicar os método para obter qualidade, é necessário 
possuir mais algumas caracteristicas para garantir a sua eficácia, como por exemplo possuir uma 
equipe que já tenha experiência com o tipo de projeto e também ter em mãos ferramentas que facilitem e 
agilizem o desenvolvimento, podendo desta forma diminuir razoavelmente o tempo de duração do projeto. 

Estão entre os exemplos de ferramentas que agilizam o trabalho do desenvolvedor: O 
\emph{\href{http://ganttproject.biz/}{Gantt Project}} \footnote{Gantt Project: 
\url{http://ganttproject.biz/}} que pode ser encontrado no site \url{http://ganttproject.biz/}, ele 
é multiplataforma, ou seja pode ser usado no \emph{Windows}, \emph{Linux} e \emph{Mac OS}, e permitem 
fazer a análise de projetos por meio de uma interface gráfica. O 
\emph{\href{http://argouml.tigris.org/}{ArgoUML}} \footnote{ArgoUML: \url{http://argouml.tigris.org/}} 
que pode ser encontrado em  \url{http://argouml.tigris.org/} também multiplataforma, permite 
por meio do \emph{\href{https://netbeans.org/}{NetBeans}} \footnote{NetBeans: 
\url{https://netbeans.org/}} gerar diagrama de classes e casos de uso, e tem a possibilidade de gerar 
códigos a partir de diagramas ou diagramas a partir de códigos. E o 
\emph{\href{http://www.eclipse.org/}{Eclipse}} \footnote{Eclipse: \url{http://www.eclipse.org/}} 
também multiplataforma que pode ser encontrado em \url{http://www.eclipse.org/}, ele tem suporte 
para várias linguagens como \emph{Java}, \emph{PHP}, \emph{C}, \emph{C++} e muito mais, e também 
possui uma grande massa de \emph{plugins} disponíveis para uso.

Além da utilização de ferramentas uma boa prática de programação utilizada por todos os 
programadores é o uso de bibliotecas que ajuda a economizar um bom tempo de desenvolvimento, 
Essa economia se deve ao fato de que com o uso desta prática, se evita de certa 
forma a re-implementação de muitas funções que já estão prontas para uso. 

As bibliotecas basicamente são funções pré-compiladas que resolvem determinados tipos de
problemas, evitando que se tenha que ‘‘invetar a roda’’ novamente, ou seja elas tem por 
finalidade disponibilizar funcinalidades que geralmente são usadas. Por exemplo 
na linguagem \emph{C} não é necessário desenvolver funções para mostrar dados na tela, basta 
incluir no projeto a biblioteca \emph{stdio} que já possui implementado a função 
\emph{printf()}, deste modo não é necessário implementar uma função para mostrar os 
dados, e sim somente fazer o uso da função \emph{printf()} contida na biblioteca \emph{stdio}.

Percebendo que a agilidade de desenvolvimento é um fator muito importante para a satisfação do 
cliente, e sabendo que o uso das bibliotecas ajuda e muito a economizar tempo. O objetivo 
deste trabalho é apresentar um tutorial de como se pode criar ou adaptar uma
biblioteca do \emph{Ruby}.

O contexto do trabalho visa ajudar uma equipe a construir e/ou adaptar
as suas própias gemas, sendo que esssas gemas possuiriam as funções que são 
mais utilizadas nos projetos que a equipe trabalha. Deste modo a equipe se preocuparia somente 
em desenvolver novas funcionalidades, pois as funcionalidades mais usuais já estariam 
presentes nas suas gemas. Assim quando fosse necessário usar algumas dessas 
funcionalidades, bastaria somente incluir as gemas necessárias no projeto e chamar as 
funções desejadas.

Estarão entre as ferramentas que veremos neste trabalho o \emph{Ruby On Rails} que é um 
\emph{framework} baseado na linguagem \emph{Ruby}, as gemas do \emph{Ruby On Rails} que 
funcionam como plugins e que podem ser incorporadas a qualquer momento no projeto, e também a 
biblioteca do \emph{Google Maps} que facilita o uso de mapas do \emph{Google}, tanto na sua 
criação como na sua manipulação.

Este trabalho esta organizado da seguinte maneira: história e conceitos de bibliotecas no 
capítulo \ref{chap:historico_e_concetios_de_bibliotecas}, onde será revisado a história 
das bibliotecas e explicado alguns conceitos básicos sobre elas, conceitos de 
\emph{Ruby} e suas bibliotecas no capítulo \ref{chap:ruby_e_suas_bibliotecas}, onde 
será apresentado alguns conceitos da linguagem \emph{Ruby}, do \emph{framework} 
\emph{Ruby On Rails} e da manipulação de suas bibliotecas (gemas), criação e adaptação
de bibliotecas do \emph{Ruby} no capítulo \ref{chap:criacao_e_adaptacao_de_bibliotecas_do_ruby}, 
onde será apresentado um tutorial de como criar e/ou adaptar uma gema, e a conclusão no 
capítulo \ref{chap:conclusao}.