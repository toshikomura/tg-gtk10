Até o ano de 2005, a rederização de mapas em uma página web custava caro, pois era necessário
servidores altamente equipados para suportar a carga de trabalho que a tarefa exigia.
Graças ao \emph{Google}, em Fevereiro de 2005, essa dor de cabeça foi aliviada com a criação
de uma nova solução de rederização, implementada na ferramenta \emph{Google Maps}. Esta nova
solução reduzia o trabalho dos servidores, possibilitando assim, a economia de dinheiro na
compra de equipamentos.

Muitas companias, percebendo a solução revolucionária adotada pelo \emph{Google} na
ferramenta \emph{Google Maps}, não ficaram atrás, adotando soluções sinilares que
também reduziam a carga de trabalho nos servidores. Entre estas ferramentas estão, o
\emph{Yahoo Maps} do \emph{Yahoo}, o \emph{Bing Maps} da \emph{Microsoft}, e o
\emph{Yandex Maps} do \emph{Yandex}.

Poara os desenvolvedores, esta nova solução de rederização, torna-se interessante
quando se analisa as operações que podem ser realizadas sobre os mapas. Por exemplo, nos mapas
do \emph{Google Maps}, podemos criar um mapa em uma página web, utilizar marcadores para
marcar locais ou regiões do mapa que consideramos importante, e até determinar caminhos
para ir de um local ao outro.

Por causa destas utilidades, vários programadores perceberam que a nova ferramenta do
\emph{Google} poderia ser muito útil em seus projeto, e por este motivo eles se mobilizaram
na procura de uma forma de incorporar o \emph{Google Maps} em suas aplicações. O
\emph{Google}, percebendo essa mobilização, resolveu facilitar o acesso a ferramenta, divulgando
em Junho de 2005 uma \emph{API} (\emph{Application Programming Interface}) para o
\emph{Google Maps}, descrevendo em detalhes como importar e utilzar as funções da ferramenta.

O \emph{Ruby On Rails} que é um \emph{framework} da linguagem \emph{Ruby}, possui
bibliotecas que simplificam o acesso da \emph{API} do \emph{Google Maps}. Basicamente estas
bibliotecas preparam internamente os objetos do mapa e depois mapeiam as funções da
ferramenta do \emph{Google}. Por exemplo, quando se esta utilizando uma destas bibliotecas,
ao tentar rederizar um mapa em uma página web, não é necessários saber quais os objetos do
\emph{Google Maps} precisam ser criados, pois estes objetos serão gerados automáticamente
na chamada da função de rederização de mapas da biblioteca.

O objetivo deste trabalho, é mostrar alguns conceitos básicos sobre bibliotecas, bem
como a criação do mesmo, apresentando um tutorial de como se pode criar e modificar
uma biblioteca do \emph{Ruby}.

O tutorial de como se pode criar um biblioteca do \emph{Ruby}, apresentará
um passo-a-paaso da implementação de uma biblioteca simples e funcional. Como exemplo,
será desenvolvido um biblioteca que faz a tradução de palavras do português para o inglês.

O tutorial de como se pode modificar uma biblioteca do \emph{Ruby}, apresentará
os passos básicos que devem ser realizados para se adaptar uma biblioteca. Neste caso, será
feito a adição da funcionalidade de desenhar caminhos entre dois locais, utilizando uma
biblioteca que mapeia a \emph{API} do \emph{Google Maps}. 

Este trabalho esta organizado da seguinte maneira: história e conceitos de
bibliotecas no capítulo \ref{chapter:historia_e_concetios_de_bibliotecas}, onde será
revisado a história das bibliotecas e explicado alguns conceitos básicos sobre elas,
bibliotecas do \emph{ruby} e segurança no capítulo \ref{chapter:bibliotecas_do_ruby_e_segurança},
onde será apresentado alguns conceitos das bibliotecas do \emph{Ruby}, criação de
bibliotecas do \emph{Ruby} no capítulo 
\ref{chapter:criacao_de_bibliotecas_do_ruby}, onde será apresentado
um tutorial de como criar uma bibliotecas do \emph{Ruby}, adaptação de
bibliotecas do \emph{Ruby} no capítulo 
\ref{chapter:adaptacao_de_bibliotecas_do_ruby}, onde será apresentado
um tutorial de como adaptar uma biblioteca do \emph{Ruby}, e a
conclusão no capítulo \ref{chapter:conclusao}.