Este capítulo tem o objetivo de apresentar brevemente os conceitos de biblioteca,
mostrar algumas definições básicas sobre as bibliotecas do \emph{Ruby}, e os riscos
e procedimento de segurança que existem para elas.

Biblioteca do inglês \emph{library}, é um conjunto de fontes de informação que possuem recursos
semelhantes. Para a computação, uma biblioteca é um conjunto de sub-programas ou
rotinas, agrupadas em um mesmo arquivo, que tem por função principal prover funcionalidades
usualmente utilizadas por desenvolvedores em um determinado contexto. Neste caso, os
desenvolvedores não precisam ter nenhum conhecimento sobre o funcionamento interno das
bibliotecas, mas precisam saber para que serve cada uma das funcionalidades e como usá-las.

Geralmente as bibliotecas possuem uma \emph{API} (\emph{Application Programming Interface}), onde
é disponibilizado as suas funcionalidades e a sua forma de uso, mostrando quais são os
parâmetros de entrada e saída, e os seus respectivos tipos.

A utilização de uma biblioteca torna-se importante, porque além de modularizar um \emph{software}, ela
permite que os desenvolvedores não se preocupem em fazer implementações repetitivas, ou seja, fazer
copias de funções de um produto para outro. Isso se deve ao fato que quando a biblioteca for
incluída em um projeto, a função já vai estar disponível para uso neste projeto.

Em seguida, neste capítulo, na seção \ref{section:bibliotecas_do_ruby}, vamos apresentar alguns
conceitos das bibliotecas do \emph{Ruby}, uma ferramenta que auxilia o acesso a estas bibliotecas,
e um modelo de segurança adotado pela comunidade.

Para mais informações sobre bibliotecas, consulte o apêndice
\ref{chapter:história_e_classificação_de_bibliotecas}, e caso ainda não conheça a linguagem
\emph{Ruby}, consulte o apêndice \ref{chapter:conceitos_e_historia_do_ruby}.


\section{Bibliotecas do Ruby}
\label{section:bibliotecas_do_ruby}

Assim como muitas linguagens, como por exemplo \emph{C}, \emph{C++}, \emph{Java}, \emph{Python} e muitas
outras, o \emph{Ruby} também possui um vasto conjunto de bibliotecas, distribuídos em 2 tipos diferentes.

A primeira forma e mais utilizada, é a distrbuição de bibliotecas na forma de \emph{gem}. O seu processo
de instalação é feito por meio do programa \emph{gem}. Por outro lado em menor número, existe a
distribuição na formata de arquivos compactados em ‘‘.zip'' ou ‘‘.tar.gz''. O seu processo de instalação
geralmente é feito por meio de arquivos de ‘‘README'' ou ‘‘INSTALL'', que possuem as instruções de
instalação [\citeonline{libraries_ruby}]. Em ambos os formatos, as bibliotecas baixadas são códigos fontes,
devido ao fato que o \emph{Ruby} é uma linguagem interpretada.

As bibliotecas do \emph{Ruby} foram apelidadas com o nome de \emph{gem} ou gemas, justamente pelo
motivo que a maior parte das bibliotecas é desenvolvida na forma de \emph{gem}.


\subsection{Programa gem}
\label{subsection:programa_gem}

O \emph{\href{https://rubygems.org/}{gem}} \footnote{gem: \url{https://rubygems.org/}} é um sistema
de pacotes do \emph{Ruby}, desenvolvido para facilitar a criação, o compartilhamento, e a instalação de
bibliotecas. Esta ferramenta possui características similares ao sistema de distribuição de pacotes
\emph{\href{https://packages.qa.debian.org/a/apt.html}{apt-get}}, no entanto ao invés de fazer a distribuição
de pacotes para \emph{\href{https://www.debian.org/}{Debian GNU/Linux distribution}} e seus variantes, ela faz
a distribuição de pacotes \emph{Ruby} por meio do servidor \emph{RubyGems} [\citeonline{libraries_ruby}].

Assim como o programa \emph{apt-get}, o programa \emph{gem} é executado em linhas de comando no formato
‘‘\emph{gem <operação> <pacotes>}'', onde ‘‘\emph{<operação>}'' pode ser ‘‘\emph{install}'' para instalar uma
gema, ‘‘\emph{search}'' para fazer buscas, entre outros. Para mais informações, consulte o apêndice
\ref{chapter:projeto_rubygems}.


\subsection{Segurança}
\label{subsection:segurança_biblioteca_ruby}

Antes de fazer a instalação de qualquer biblioteca em qualquer linguagem é necessário verificar
se a fonte é confiável. Isso não é diferente na linguagem \emph{Ruby}, pois a todo momento podemos
abrir brechas e com isso correndo riscos de perder informações.

As bibliotecas do \emph{Ruby} possuem como repositório padrão o \emph{RubyGems}. Deste modo, fica claro
que existe uma grande quantidade de bibliotecas neste repositório, e também que é quase impossível
garantir que 100\%  destas bibliotecas sejam confiáveis.

A ferramenta \emph{gem} que faz a instalação das gemas no ambiente de desenvolvimento, possuí uma política
de segurança que tenta garantir que as bibliotecas instaladas são de uma fonte de confiança. Esta política
funciona por meio da verificação de certificados digitais que consiste em um processo de 2 passos. 

No primeiro passo, depois da criação da biblioteca, pessoas ou orgãos validam a confiabilidade da gema por
meio de verificações e testes de segurança. Após esta verificação, caso a biblioteca seja confiável, ela
recebe uma assinatura, feita por meio de uma chave privada de um certificado, para comprovar a sua segurança.
Deste modo, todas as bibliotecas que possuem essa assinatura, são confiáveis, pois pela assinatura sabemos
que determinada pessoa ou orgão comprovaram que a biblioteca é segura.

No segundo passo, o usuário final no momento da instalação da biblioteca, requisita por meio da política
de segurança do \emph{RubyGems}, a validação de confiabilidade da biblioteca a ser instalada. Existem vários
níveis de segurança. O mais baixa, é a ‘‘\emph{NoSecurity}'' que não faz nenhuma verificação no momento da
instalação. O mais alta, é a política ‘‘\emph{HighSecurity}'' que só permite a instalação de gemas que
estão assinadas e sem nenhuma modificação. Neste segundo caso, o \emph{RubyGems} ao receber essa
requisição, verifica por meio do certificado e da assinatura da biblioteca, se ela recebeu a assinatura
de alguma pessoa ou orgão de confiança, e se ela não foi modificada após a sua assinatura, pois
modificação podem conter códigos maliciosos.

Apesar deste método de chaves criptografadas ser benéfico, ele geralmente não é usado, pois é necessário
vários passos manuais no desenvolvimento, e também não existe nenhuma medida de confiança bem definida
para estas chaves de assinatura [\citeonline{guide_security_rubygems}].

O \emph{RubyGems} possui um esquema para reportar vulnerabilidades das bibliotecas do \emph{Ruby}.
Este esquema é constituído de dois tipos. Em um tipo se reporta erros na gema de outros usuários e no outro
se reporta erros da própia gema. O esquema para cada tipo, será explicado a seguir.

Para reportar vulnerabilidade de \emph{gemas} de outros usuários, sempre é necessário verificar
se a vulnerabilidade ainda não é conhecida. Caso ela ainda não seja conhecida, é recomendado que se
reporte o erro por um e-mail privado, diretamente para o dono da gema, informando o problema e
indicando uma possível solução.

Por outro lado para uma vulnerabilidade na própia \emph{gema}, é necessário que se requisite um
identificador \emph{\href{https://cve.mitre.org/}{CVE}}
\footnote{CVE: \url{https://cve.mitre.org/}} via e-mail para
\href{mailto:cve-assign@mitre.org}{\nolinkurl{cve-assign@mitre.org} }, pois deste modo, existirá
um identificador único para o problema. Com o identificador em mãos, é necessário trabalhar em uma
possível solução. Assim que encontrar uma solução, será necessário criar um \emph{patch} de correção.
E finalmente depois de criar o \emph{patch}, deve-se informar a comunidade que existia um problema na
\emph{gema} e que essa vulnerabilidade foi corrigida no \emph{patch ‘‘x''}.

Para este segundo caso, também recomenda-se adicionar o problema em um \emph{database open source}
de vulnerabilidade, como por exemplo o \href{http://osvdb.org/}{OSVDB}
\footnote{OSVDB: \url{http://osvdb.org/}}. Além disso, também recomenda-se enviar um e-mail para
\href{mailto:ruby-talk@ruby-lang.org} {\nolinkurl{ruby-talk@ruby-lang.org} } com o
\emph{subject: ‘‘[ANN][Security]''}, informando detalhes sobre a vulnerabilidade, as versões que
possuem o erro, e as ações que devem ser tomadas para corrigir o problema.
