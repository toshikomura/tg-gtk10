Este capítulo tem o objetivo de mostrar detalhadamente os códigos \emph{Ruby} que foram mostrados na forma
de algoritmo no trabalho.

\lstinputlisting[ style=customRuby, caption={translatetoenglish.rb}, label={lst:translatetoenglish.rb}]
{codigos/translatetoenglish_simplificado.rb}

O arquivo mostrado no código \ref{lst:translatetoenglish.rb}, explicado logo a seguir. faz referência
ao algoritmo mostrado no código \ref{lst:algoritmo_de_translatetoenglish.rb}.

\begin{itemize}

  \item ‘‘\emph{module ... end} nas linhas ‘‘1'', ‘‘2'' e ‘‘3'' define a árvore de módulos
  ‘‘\emph{Gemtranslatetoenglish}'' na raiz, ‘‘\emph{Helpers}'' no segundo nível e ‘‘\emph{Translatetoenglish}'' no
  terceiro.

  \item ‘‘\emph{def translate( phrase) ... end}'' da linha ‘‘5'' até a linha ‘‘34'', define
  a função de tradução da gema, onde foi definido somente duas traduções, ‘‘\emph{OLÁ}''
  para ‘‘\emph{HELLO}'' e ‘‘MUNDO'' para ‘‘\emph{WORLD}''.

  A árvore definida no código \ref{lst:translatetoenglish.rb} foi necessário por causa do código
  inserido na linha ‘‘8'' ‘‘\emph{ActionController::Base.helper Gemtranslatetoenglish::Helpers::Translatetoenglish}''
  no código \ref{lst:gemtranslatetoenglish.rb} que serve para evitar a necessidade de escrever a
  \emph{PATH} completa na \emph{view} para chamar uma função da gema na \emph{view}.

\end{itemize}


O arquivo mostrado no código \ref{lst:funcoes_adicionais_do_handler}, explicado logo a seguir, faz
referência ao algoritmo mostrado no código \ref{lst:algoritmo_de_funcoes_adicionais_do_handler}.

\begin{itemize}

 \item Na linha ‘‘2'' é adicionado a função ‘‘\emph{addDirection}'' que recebe como parâmetro
 ‘‘\emph{direction\_data}'', que possui o informações do local de origem e local de destino que são
 obrigatórios para criar a direção, e ‘‘\emph{provider\_options}'' que pode conter as opções da forma
 como esse direção deve ser gerada. Essa função tem por objetivo criar as direções e colocá-las no mapas.

 \item Na linha ‘‘13'' é adicionado a função ‘‘\emph{calculate\_route}'' com o parâmetro
 ‘‘\emph{direction\_data}'', que tem por principal objetivo fazer a requisição para o \emph{Google } de uma
 possível direção entre o local de origem ao local de destino.

 \item Da linha ‘‘3'' a linha ‘‘10'' é o conteúdo da função ‘‘\emph{addDirection}'', onde se cria os
 atributos ‘‘\emph{direction\_service}'' e ‘‘\emph{direction\_render}'' para o ‘‘\emph{Handler}''. Para
 cada um deste atributos, é atribuindo o seu respectivo objeto do \emph{Google Maps} com a chamada da função
 ‘‘\emph{@\_builder(...)}''. Depois é feito a chamada da função ‘‘\emph{calulate\_route(...)}''
 para encontrar uma possível direção. E finalmente com o código
 ‘‘\emph{@direction\_render.getServiceObject().setMap(@getMap())}'', a direção encontrada é colocado
 no mapa.

 \item Da linha ‘‘13'' a linha ‘‘20'' é o conteúdo da função ‘‘\emph{calculate\_route(...)}'', onde
 inicialmente é colocado em uma variável local o \emph{status Ok} de requisição que será utilizado
 para verificar se a requisição de direção foi feita com sucesso, e também é criada uma variável local
 para o \emph{objeto DirectionRender} do \emph{Google Maps}. Em seguida, através da chamada da função
 ‘‘\emph{route(...)}'' do \emph{objeto DirectionService}. Nesta função so parâmetros de entrada são,
 o local de origem e o local de destino, e como resposta se recebe o \emph{status} da requisição
 e o \emph{response}. Neste caso, \emph{response} pode ou não conter o caminho solução. Após a execução
 desta função, caso a resposta venha com o \emph{status Ok}, o \emph{objeto DirectionRender} recebe o
 caminho solução.

\end{itemize}

\lstinputlisting[ style=customCoffee, caption={Funções adicionais do Handler}, label={lst:funcoes_adicionais_do_handler}]
{codigos/funcoes_adicionais_do_handler.coffee}
