Neste trabalho apresentamos alguns conceitos básicos sobre bibliotecas, descrevendo a imprtância de seu
uso no capitulo \ref{chap:historico_e_concetios_de_bibliotecas}. Depois dedicamos um pouco do tempo
no capítulo \ref{chap:ruby_e_suas_bibliotecas} para falar sobre a linguagem \emph{Ruby}, suas
bibliotecas, e a possibilidade de se utilzar chaves para se ter segurança na instalação destas bibliotecas.

Percebendo que somente a utilização de bibliotecas de terceiros, as vezes não é o suficiente para se
ter economia de tempo, passamos para um outro nível, aonde possibilitamos uma equipe ou mesmo somente
uma pessoa a desenvolver a sua própia biblioteca quando uma certa funcionalidade é muito utlizada em vários
tipos de projetos.

Contudo no capítulo \ref{chap:criacao_e_adaptacao_de_bibliotecas_do_ruby} apresentamos um tutorial
básico de como se pode criar ou modificar uma biblioteca do \emph{Ruby}, descrevendo em detalhes os
passos de implementação que devem ser seguidos para se ter mais chances de conseguir sucesso no
projeto. Neste capítulos também detalhamos as ferramentas e os comandos utilizados durante o processo de
desenvolvimento apresentando exemplos para facilitar a compreensão do tutorial.

Acreditamos que com a apresentação deste trabalho, uma equipe dependendo das suas necessidades tem
a possibilidade de desenvolver do zero ou adaptar bibliotecas do \emph{Ruby}, podendo assim
economizar tempo de desenvolvimento em projetos futuros.

Para este trabalho no exemplo de criação de \emph{gemas}, desenvolvemos uma \emph{gema} simples de
tradução, mostrando somente alguns detalhes da linguagem \emph{Ruby}, e por esse motivo para
trabalhos futuros, poderia ser criada uma nova biblioteca ou mesmo fazer uma adaptação da
\emph{gema} de exemplo para mostrar mais conceitos da linguagem. Também para a biblioteca
\emph{Google-Maps-For-Rails} adaptada para aceitar a funcionalidade de gerar direções, poderia ser feito o
incremento de mais funcionalidades, como por exemplo, a inclusão da escolha do tipo de locomoção a ser
utilizada no percurso e a possibilidade de adicionar locais intermediários no caminho.
