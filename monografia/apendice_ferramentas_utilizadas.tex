Este capítulo tem o objetivo de apresentar as ferramentas que foram utilizadas no tutorial. Na seção
\ref{section:vmware_player}, apresentaremos a ferramenta \emph{VMware® Player}, em seguida na seção
\ref{section:rvm}, apresentaremos a ferramenta \emph{RVM}, depois na seção \ref{section:ruby_on_rails},
apresentaremos o \emph{framework Ruby On Rails}, e por fim na seção \ref{section:git}, apresentaremos
a ferramenta \emph{git}.

\section{VMware® Player}
\label{section:vmware_player}

\emph{\href{http://www.vmware.com/products/player}{VMware® Player}} é um aplicativo de virtualização de 
\emph{desktop} que pode rodar um ou mais sistemas operacionais ao mesmo tempo no mesmo computador sem 
a necessidade de reiniciar a máquina. Possui uma interface fácil de usar, suporte a vários sistemas 
operacionais, como por exemplo \emph{Windows}, \emph{MAC} e \emph{Linux} e é portável 
[\citeonline{vmplayer}].

Com o \emph{VMware® Player} pode-se fazer o compartilhamento de vários recursos, como por exemplo pode-se 
fazer a transferência de arquivos do sistema opercional virtual com o sistema opercional que está rodando 
na máquina fisica. 

\section{RVM}
\label{section:rvm}

\emph{\href{http://rvm.io/}{Ruby Version Manager}} \footnote{RVM: \url{http://rvm.io/}} é uma 
ferramenta de linha de comando que permite facilmente instalar, gerenciar e trabalhar com multiplos ambientes 
do \emph{Ruby} para interpretar um certo conjunto de gemas [\citeonline{rvm}].

 \href{https://github.com/wayneeseguin}{Wayne E. Seguin} \footnote{Wayne E. Seguin: 
 \url{https://github.com/wayneeseguin}} iniciou o projeto do \emph{RVM} em outubro de 2007
e a partir de então com uma considerável experiência programando em \emph{Bash}, \emph{Ruby} e outras 
linguagens obteve o conhecimento suficiente para criar o gerenciador [\citeonline{about_rvm}].

\section{Ruby On Rails}
\label{section:ruby_on_rails}

\emph{\href{http://rubyonrails.org/}{Ruby On Rails}} \footnote{Ruby On Rails: \url{http://rubyonrails.org/}} 
é um \emph{web framework} de código aberto que tem por finalidade facilitar a programação
visando a produtividade sustentável. Este \emph{framework} permite escrever códigos bem estruturados
favorecendo a manutenção de aplicações [\citeonline{ruby_on_rails}].

O \emph{Rails} foi criado em 2003 por \emph{\href{http://david.heinemeierhansson.com/}{David Heinemeier Hansson}}
\footnote{David Heinemeier Hansson: \url{http://david.heinemeierhansson.com/}} e desde então é extendido 
pela equipe do \emph{\href{http://rubyonrails.org/core/}{Rails Core Team}} 
\footnote{Rails Core Team: \url{http://rubyonrails.org/core/}} e mais de 3.400 usuários 
[\citeonline{ruby_on_rails}].

\section{Git}
\label{section:git}

\emph{\href{http://git-scm.com/}{Git}} \footnote{Git: \url{http://git-scm.com/}} é um sistema distribuído de 
controle de versão livre e de código aberto, desenhado para controlar projetos pequenos e grandes com 
rapidez e eficiência. É uma ferramenta fácil de manipular com alto desempenho [\citeonline{git}].

Por irônia do destino o \emph{git} foi criado em 2005 por 
\emph{\href{http://torvalds-family.blogspot.com.br/}{Linus Torvalds}} 
\footnote{Linus Torvalds: \url{http://torvalds-family.blogspot.com.br/}} graças ao fim da relação entre a 
comunidade que desenvolvia o \emph{Linux} e a compania que desenvolvia o 
\emph{\href{http://www.bitkeeper.com/}{BitKeeper}}, ferramenta que até aquele ano fazia o gerênciamento de 
código do \emph{Linux}. Sem uma ferramento \emph{SCM} (\emph{Source Code Management}), \emph{Torvalds} 
resolveu desenvolver uma ferramenta parecida com o \emph{BitKeeper} que possuiria mais velocidade,
\emph{design} simples, grande suporte para \emph{non-linear development} (centenas de \emph{branchs}), 
completamente distribuído e capacidade de controlar grandes projeto com grandes quantidades de dados.