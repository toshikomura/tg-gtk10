Este capitulo tem o objetivo de mostrar alguns conceitos básicos sobre o \emph{Ruby}, suas
história, suas bibliotecas, riscos e procedimento de segurança que existem.
Na seção \ref{section:conceitos_ruby} iremos apresentar alguns conceitos da linguagem \emph{Ruby},
depois vamos falar um pouco sobre sua história na seção \ref{section:história_ruby}, depois vamos ver
a importância de uma \emph{API} na seção \ref{section:API_ruby}, e por fim na seção
\ref{section:segurança_ruby}, vamos falar um pouco sobre a segurança do \emph{Ruby}.

\section{Conceitos}
\label{section:conceitos_ruby}

\emph{\href{https://www.ruby-lang.org/en/}{Ruby}} é uma linguagem de programação
dinâmica de código aberto com foco na simplicidade e produtividade, tem uma sintaxe elegante, é natural de
ler e escrever, e foi inventada por \emph{ \href{http://www.rubyist.net/~matz/}{Yukihiro ‘‘Matz'' Matsumoto}}
\footnote{Yukihiro ‘‘Matz'' Matsumoto: \url{http://www.rubyist.net/~matz/}}
que tentou cuidadosamente criar uma linguagem balanceada, misturando as suas linguagens favoritas
( \emph{Perl}, \emph{Smalltalk}, \emph{Eiffel}, \emph{Ada}, e \emph{Lisp}) [\citeonline{ruby_site}].

\emph{{Yukihiro ‘‘Matz'' Matsumoto}} sempre enfatiza que ele estava ‘‘tentando fazer o \emph{Ruby} natural, não simples'' e também com toda a
sua experiência acrescenta que ‘‘\emph{Ruby} é simples na aparência, mas é muito complexo internamente, assim
como o corpo humano.''. \emph{‘‘Matz''} também relembra que procurava uma linguagem de script que fosse mais
poderosa que \emph{Perl} e mais orientada a objeto do que \emph{Python}, tentando encontrar a
sintaxe ideal nas outras linguagens [\citeonline{matz_ruby_talk_main_list}] [\citeonline{matz_often_sya_of_ruby}].

Em muitas linguagens números e outros tipos primitivos não são \emph{objetos}, no entanto no \emph{Ruby}
cada pedaço de informação possui propriedades e ações, ou seja, tudo é \emph{objeto}, Neste caso o
\emph{Ruby} possui esse tipo de característica, pois ele seguiu a influência do \emph{SmallTalk}, onde
todos seus tipos, inclusive os mais primitivos como inteiros e booleanos são objetos.

Logo abaixo no código \emph{Ruby} \ref{lst:objeto_ruby} segue um
exemplo onde construimos um \emph{método} ‘‘tipo'' para o \emph{objeto} inteiro ‘‘5''. Suponde que se
requisite o \emph{método} ‘‘\emph{5.tipo}'' o retorno seria ‘‘\emph{Tipo inteiro}''.

\lstinputlisting[ style=customRuby, caption={Tudo é objeto para o Ruby}, label={lst:objeto_ruby}]
{codigos/objeto_ruby.rb}

O \emph{Ruby} também é flexível, pois possibilita remover ou redefinir seu própio \emph{core}, por
exemplo no código \ref{lst:ruby_flexivel} acrescentamos na \emph{classe ‘‘Numeric''}
o \emph{método} ‘‘vezes()'' que possui a mesma característica do operador ‘‘*''. Ao se fazer a execução
deste código a variável ‘‘y'' recebe o valor ‘‘10'' como resultado da operação ‘‘5.vezes 2''.

\lstinputlisting[ style=customRuby, caption={Ruby flexível}, label={lst:ruby_flexivel}]
{codigos/ruby_flexivel.rb}

Apesar de ser flexível esta linguagem possui algumas convenções com por exemplo no código
\ref{lst:convencao_ruby}, a definição de ‘‘var'' ou qualquer outro nome de
variável dependendo do contexto sozinha deve ser uma variável local, ‘‘@var'' deve ser uma instância
de uma variável por causa do caracter ‘‘@'' como primeiro caracter e ‘‘\$var'' deve ser uma variável global
por causa do caracter ‘‘\$'' como primeiro caracter.

\lstinputlisting[ style=customRuby, caption={Convenção Ruby}, label={lst:convencao_ruby}]
{codigos/convencao_ruby.rb}

\section{História}
\label{section:história_ruby}

O \emph{Ruby} foi criado em 24 de fevereiro de 1993 e o com o passar do tempo foi ganhando espaço na
comunidade de desenvolvedores, chegando em 2006 a ter uma grande massa de aceitação, possuindo várias
conferências de grupos de usuários ativos pelas principais cidades do mundo [\citeonline{ruby_site}].

O nome \emph{Ruby} veio a partir de uma conversa de chat entre \emph{Matsumoto} e \emph{Keiju Ishitsuka},
antes mesmo de qualquer linha de código ser escrita. Inicialmente dois nomes foram propostos ‘‘\emph{Coral}''
e ‘‘\emph{Ruby}''. \emph{Matsumoto} escolheu o segundo em um e-mail mais tarde. E depois de um tempo, após
já ter escolhido o nome, ele percebeu no fato de que \emph{Ruby} era o nome da \emph{birthstone} (pedra
preciosa que simboliza o mês de nascimento) de um de seus colegas.

Sua primeira versão foi anunciada em 21 de dezembro de 1995 na ‘‘\emph{Japanese domestic newsgroups}''.
Subsequêncialmente em dois dias foram lançadas mais três versões juntamente com a
‘‘\emph{Japanese-Language ruby-list main-list}'' (\emph{RubyTalk}).

\emph{Ruby-Talk} \footnote{Ruby-Talk: \url{https://www.ruby-forum.com/}} a primeira \emph{lista de
discussão da Ruby} \footnote{Listas de discussões do Ruby:
\url{https://www.ruby-lang.org/en/community/mailing-lists/}} chegou a possuir em média 200 mensagens por dia
em 2006. E com o passar dos útlimos anos, essa média veio a cair, pois o crescimento da comunidade
obrigou a criação de grupos específicos, empurrando os usuários da lista principal para as lista específica.


\section{API}
\label{section:API_ruby}

Uma \emph{Application Programming Interface (API)} possui como principal função definir de forma simplificada
o acesso as funcionalidades de um certo componente, informando ao usuário somente para que serve cada função
e como usá-la, indicando os parâmetros de entrada e saída com os seus respectivos tipos.

Basicamente uma \emph{API} diminui a complexidade para o seu usuário. Por exemplo por analogia a \emph{API}
de um elevador seria o seu painel, e a entrada seria o andar que se deseja ir. Desta forma supondo que
um usuário deseja ir do segundo para o décimo andar, bastaria ele apertar o botão para chamar o
elevador. Depois quando o elevador chegasse bastaria ele entrar e apertar o botão (10), e mesmo sem ter
nenhum conhecimento mecânico do elevador, ele conseguiria obter como saída o décimo andar.

A \emph{API} do \emph{Ruby} não é diferente, ou seja, não é necessário que se tenha
conhecimento do funcionamento do \emph{core} do \emph{Ruby}
\footnote{ Core do Ruby: \url{http://www.ruby-doc.org/core-2.1.3/}}, basta saber para que
serve cada uma de suas funções disponibilizadas, seus parâmetro de entrada e saída com seus respectivos tipos.

\section{Segurança}
\label{section:segurança_ruby}

A proteção de dados sempre foi uma questão muito discutida e na comunidade do \emph{Ruby} isso não é
diferente, justamente porque a todo momento estamos sujeitos a sofrer ataques, e com isso podemos ser
prejudicados, perdendo dinheiro e informações sigilosas.

A comunidade do \emph{Ruby}, sabendo destes problemas, construiu um esquema para corrigir as
vulnerabilidades de segurança. Neste esquema, as vulnerabilidade descobertas são reportadas via e-mail para
\href{mailto:security@ruby-lang.org}{\nolinkurl{security@ruby-lang.org}}, que é uma lista privada com
membros que administram o \emph{Ruby}, como por exemplo \emph{Ruby commiters}. Neste esquema, os membros da
lista de segurança, por medidas de segurança, somente compartilham as vulnerabilidades quando elas já estão
solucionadas. Na publicação da vulnerabilidade é informado o tipo de erro, os problemas que o erro causa,
e a solução que deve ser tomada para sua correção [\citeonline{security_ruby}].

Um exemplo é a vulnerabilidade publicada em 10/04/2014, referênciado com o identificador
\emph{\href{https://cve.mitre.org/}{CVE}}
(\href{https://web.nvd.nist.gov/view/vuln/detail?vulnId=CVE-2014-0160}{CVE-2014-0160)}, fala sobre um grave
problema na implementação no \emph{\href{https://www.openssl.org}{OpenSSL’s}} do \emph{TLS/DTLS} (\emph{transport
layer security protocols}). Neste problema, uma pessoa mal intencionada poderia roubar dados da memória, tanto
na comunicação entre o servidor com o cliente, como na comunicação do cliente para o servidor
[\citeonline{openssl_problem_security_ruby}].