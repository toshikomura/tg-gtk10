Biblioteca do inglês \emph{library}, é um conjunto de fontes de informação que possuem recursos 
semelhantes. Para a computação, uma biblioteca é um conjunto de subprogramas ou 
rotinas que tem por função principal prover funcinalidades usualmente utilizadas por 
desenvolvedores em um determinado contexto. Neste caso os desenvolvedores não precisam ter 
nenhum conhecimento sobre o funcionamento interno das bibliotecas, mas precisam saber para que 
serve cada uma destas funcionalidades e como se deve usá-las. 

Geralmente as bibliotecas possuem uma \emph{API} (\emph{Application Programming Interface}), onde 
é disponiblizado as suas funcionalidades e a sua forma de uso, mostrando quais são os seus 
parâmetros de entrada e saída e os seus respectivos tipos.

A utilização de uma biblioteca torna-se importante, porque além de modularizar um \emph{software}, ela
permite que os desenvolvedores não se preocupem em fazer implementações repetitivas, ou seja, fazer 
copias de funções de um produto para outro, e isso se deve ao fato de que quando a biblioteca for 
incluída no projeto, a função já vai estar disponível para uso.

Neste capítulo veremos um pouco sobre a história das bibliotecas na seção 
\ref{section:historia_biblioteca} e depois veremos as possíveis classificações de bibliotecas na seção 
\ref{section:classificação_biblioteca}.

\section{História}
\label{section:historia_biblioteca}

Os primeiros conceitos que tinham proximidades com a definição de bibliotecas apareceram publicamente 
com o \emph{software} ‘‘\emph{COMPOOL}'' (\emph{Communication Pool}) desenvolvido em \emph{JOVIAL}, 
que é uma linguagem de alto nível parecido com \emph{ALGOL}, mas especializada para desenvolvimento 
de sistemas embarcados. O ‘‘\emph{COMPOOL}'' tinha como propósito compartilhar os dados do 
sistema entre vários progrmas, fornecendo assim informação centralizada. Com essa visão o 
‘‘\emph{COMPOOL}'' seguiu os princípios da ciência da computação, separando interesses e 
escondendo informações [\citeonline{history_of_programming_languages}].

As linguagens \emph{COBOL} e \emph{FORTRAN} também possuiam uma prévia implementação do sistema de
bibliotecas. O \emph{COBOL} em 1959 tinha a capacidade de comportar um sistema primitivo de 
bibliotecas, mas segundo \emph{Jean Sammet} esses sistema era inadequado. Já o 
\emph{FORTRAN} possuia um sistema mais moderno, onde ele permitia que os subprogramas poderiam ser 
compilados de forma independente um dos outros, mas com essa nova funcionalidade o compilador acabou 
ficando mais fraco com relação a ligação, pois com essa possibilidade adicionada ele não conseguia 
fazer a verificação de tipos entre os subprogramas [\citeonline{history_of_programming_languages}].

Por fim chegando no ano de 1965 com a linguagem \emph{Simula 67}, que foi a primeira linguagem
de programação orientada a objetos que permitia a inclusão de suas classes em arquivos de bibliotecas.
Ela também permitia que os arquivos de bibliotecas fossem utilizadas em tempo de compilação para 
complementar outros programas [\citeonline{history_of_programming_languages}].

\section{Classificação}
\label{section:classificação_biblioteca}

As bibliotecas podem ser indentificadas pelo seu tipo de classificação, onde as suas classificações 
podem ser definidas pela maneira como elas são ligadas que é explicado na sub-seção 
\ref{subsection:formas_de_ligamento}, pelo momento que elas são ligadas que é explicado na sub-seção 
\ref{subsection:momentos_de_ligação}, e pela forma como elas são compartilhadas que é explicado na
sub-seção \ref{subsection:formas_de_compartilhamento}.

\subsection{Formas de ligamento}
\label{subsection:formas_de_ligamento}

O processo de ligamento implica em associar uma biblioteca a um programa ou outra biblioteca, ou seja,
é nesse momento que as implementações das funcionalidades são realmentes ligadas ao programa ou
outra biblioteca. 

Esse processo de ligamento pode ser feito de 3 formas diferentes:

\begin{itemize}

 \item \emph{\textbf{Tradicional}} que significa que os dados da biblioteca são copiados para o executável do programa
 ou outra biblioteca.
 
 \item \emph{\textbf{Dinâmica}} que significa que ao invés de copiar os dados da biblioteca para o executável, é 
 somente feito uma referência do arquivo da biblioteca. Neste caso o ligador não tem tanto 
 trabalho na compilação, pois ele somente grava a biblioteca a ser utilizada e um indice para 
 ela, passando todo o trabalho para o momento onde a aplicação é carregada para a memória ou para 
 o momento que a aplicação requisita a biblioteca.
 
 \item \emph{\textbf{Remoto}} que significa que a biblioteca vai ser carregada por chamadas de procedimento remotos, 
 ou seja, a biblioteca pode ser carregada mesmo não estando na mesma máquina do programa ou 
 biblioteca, pois ela é carregada pela rede.
 
\end{itemize}

\subsection{Momentos de ligação}
\label{subsection:momentos_de_ligação}

O momento de ligação diz respeito ao momento em que a biblioteca vai ser carregada na memória e 
para esse caso existem 2 formas:

\begin{itemize}

 \item \emph{\textbf{Carregamento em tempo de carregamento}} que significa que a biblioteca vai ser 
 carregada na memória quando a aplicação também estiver sendo carregada.
 
 \item \emph{\textbf{Carregamento dinâmico ou atrasado}} que signfica que a biblioteca só vai ser carregada na 
 memória quando a aplicação requisitar o seu carregamento e isso é feito em tempo de execução.
 
\end{itemize}

\subsection{Formas de compartilhamento}
\label{subsection:formas_de_compartilhamento}

As formas de compartilhamento de bibliotecas se divide em 2 conceitos. Sendo que o primeiro se refere 
ao compartilhamento de código em disco entre os vários programas. E o segundo se refere ao 
compartilhamento da biblioteca na memória em tempo de execução. 

O compartilhamento em mémória traz a vantagem de que dois ou mais programas podem 
compartilhar o acesso ao mesmo código da biblioteca na memória, com isso se evita que a mesma 
biblioteca seja colocada mais de uma vez na memória. 

Por exemplo para o segundo conceito, suponha que os programas \emph{P1} e \emph{P2} necessitem de 
uma biblioteca \emph{B} que é carregada no \emph{momento de carregamento}. Suponha 
agora que executamos o programa \emph{P1} e depois de um tempo executamos o programa \emph{P2}, desta 
forma primeiramente o programa \emph{P1} e a biblioteca \emph{B} são carregadas na memória. Como 
também executamos o programa \emph{P2}, ele também é carregado na memória, mas sem a necessidade de 
carregar a biblioteca \emph{B}, pois ela já havia sido carregada anteriomente.

Pensando pelo outro lado, o compartilhamento de memória pode ser um pouco prejudicial 
ao desempenho, pois essa biblioteca deve ser escrita para executar em um ambiente \emph{multi-tarefa} 
e isso pode causar alguns atrasos.