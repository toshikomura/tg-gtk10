O projeto \emph{RubyGems} em conjunto com a ferramenta \emph{gem}, foi criado em Abril de 2009 por
\emph{\href{https://twitter.com/qrush}{Nick Quaranto}} \footnote{Nick Quaranto: \url{https://twitter.com/qrush}},
numa tentativa de proporcionar uma melhor \emph{API} para acessar as gemas [\citeonline{about_rubygems}].

Com o passar do tempo, o projeto cresceu, chegando a possuir mais de 115 \emph{Rubyistis} contribuintes e
milhões de gemas baixadas do \emph{RubyGems} através da ferramenta [\citeonline{about_rubygems}].

Até a versão ‘‘\emph{1.3.6}'', o nome do projeto era \emph{Gemcutter}, sendo renomeada a partir desta versão
para \emph{RubyGems}, objetivando a solidificação do papel central do site na comunidade do \emph{Ruby}
[\citeonline{about_rubygems}].

A instalação do \emph{gem} pode ser feita pelo terminar, executando ‘‘\emph{sudo apt-get install gem}'' com
privilégios de administrador ou acessando o site do projeto
\emph{\href{RubyGems}{https://rubygems.org/pages/download/}} e baixando a última
versão do \emph{gem}. Para o caso de baixar a instalação, deve-se descompactar o pacote, entrar
no diretório e executar ‘‘\emph{ruby setup.rb}'' com privilégios de administrador.

Caso haja uma versão do \emph{gem} instalada, pode-se fazer a atualização para a última versão executando
‘‘\emph{gem update --system}'' com privilégios de administrador.

Se ocorrer algum problema no momento da instalação, atualização ou se necessitar de mais
alguma informações, se pode executar ‘‘\emph{ruby setup.rb --help}'' para obter ajuda.

A partir da versão ‘‘\emph{1.9}'' do \emph{Ruby} esse processo de instalação não é mais necessário, pois o
\emph{gem} vem por \emph{default} instalado junto com \emph{Ruby}, mas para as versões anteriores
é necessário fazer a instalação manualmente.