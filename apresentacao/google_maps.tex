\subsection{Processo de Adaptação}
\begin{frame}
 \frametitle{Processo de Adaptação}

  \begin{block}{Processo}

    \begin{enumerate}

     \item Realizar \emph{\textbf{engenharia reversa}} para gerar diagramas de alto nível.

     \item Entender a biblioteca por meio dos diagramas de alto nível.

     \item Realizar adaptações.

    \end{enumerate}

  \end{block}
  
\end{frame}


\subsection{Funcionalidades Google-Maps-For-Rails}
\begin{frame}
 \frametitle{Funcionalidades Google-Maps-For-Rails}
  
  \begin{block}{API Google Maps e Gema Google-Maps-For-Rails}       
  
    \begin{itemize}        
    
     \item \emph{\textbf{Google-Maps-For-Rails}} mapeia \emph{\textbf{API}} do \emph{\textbf{Google Maps}}.
     
     \item \emph{\textbf{API Google Maps}} utliza a linguagem \emph{\textbf{Javascript}}.     

     \item \emph{\textbf{Google-Maps-For-Rails}} utiliza \emph{\textbf{CoffeeScript}} -$>$ \emph{\textbf{Javascript}}.     

    \end{itemize}

  \end{block}    
 
  \begin{block}{Funcionalidades}   

   \begin{itemize}

    \item  Cria mapas do \emph{\textbf{Google Maps}} em páginas web.

    \item  Permite a criação de mais de um mapa na mesma página (id).

    \item  Cria sobreposições para os mapas (markers, circles, polygon, polylines, etc).

    \item  Permite a criação de vários sobreposições do mesmo tipo com a chamada de uma única função.    

    \item  Permite readaptação do \emph{\textbf{core}} (\emph{\textbf{Bing Maps}}).

   \end{itemize}

  \end{block}

\end{frame}


\subsection{API Google-Maps-For-Rails}
\begin{frame}
 \frametitle{API Google-Maps-For-Rails}

      \lstinputlisting[ style=customCoffee, caption={Exemplo CoffeeScript API Google-Maps-For-Rails}, label={lst:exemplo_coffeescript_api_google-maps-for-rails}]
      {../monografia/codigos/exemplo_coffeescript_api_google-maps-for-rails_apresentacao.js.coffee}

\end{frame}


 \subsection{Proposta de Novas Funcionalidades}
\begin{frame}
 \frametitle{Proposta de Novas Funcionalidades}

  \begin{block}{Novas Funcionalidades}   

   \begin{itemize}

    \item  Gerar direções entre dois lugares no mapa.    

    \item  Gerar direções entre dois lugares com pontos intermediários obrigatórios.

    \item  Permitir a troca de cores entre os tracejados entre dois pontos do caminho.
    
    \item  Permitir a escolha do veículo de locomoção.    
    
    \item  Mostrar detalhes sobre o caminho (tempo de percurso, distância, etc).

   \end{itemize}

  \end{block}

\end{frame}


\subsection{API Google-Maps-For-Rails Adaptada}
\begin{frame}
 \frametitle{API Google-Maps-For-Rails Adaptada}

  \begin{block}{Criar Direções}

   Por meio de código \emph{\textbf{CoffeeScript}}:

      \lstinputlisting[ style=customCoffee, caption={Exemplo CoffeeScript API Google-Maps-For-Rails Adaptado}, label={lst:exemplo_coffeescript_api_google-maps-for-rails_adaptado}]
      {../monografia/codigos/exemplo_coffeescript_api_google-maps-for-rails_adaptado.js.coffee}

  \end{block}

\end{frame}
