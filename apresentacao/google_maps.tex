\subsection{API Google-Maps-For-Rails}
\begin{frame}
 \frametitle{API Google-Maps-For-Rails}

      \lstinputlisting[ style=customCoffee, caption={Exemplo CoffeeScript API Google-Maps-For-Rails}, label={lst:exemplo_coffeescript_api_google-maps-for-rails}]
      {../monografia/codigos/exemplo_coffeescript_api_google-maps-for-rails.js.coffee}

\end{frame}


\subsection{Processo de Adaptação}
\begin{frame}
 \frametitle{Processo de Adaptação}

  \begin{block}{Processo}

    \begin{enumerate}

     \item Realizar \emph{\textbf{engenharia reversa}} para gerar diagramas de alto nível.

     \item Entender a biblioteca por meio dos diagramas de alto nível.

     \item Realizar adaptações.

    \end{enumerate}

  \end{block}

\end{frame}


\subsection{API Google-Maps-For-Rails Adaptada}
\begin{frame}
 \frametitle{API Google-Maps-For-Rails Adaptada}

  \begin{block}{Criar Direções}

   Por meio de código \emph{\textbf{CoffeeScript}}:

      \lstinputlisting[ style=customCoffee, caption={Exemplo CoffeeScript API Google-Maps-For-Rails Adaptado}, label={lst:exemplo_coffeescript_api_google-maps-for-rails_adaptado}]
      {../monografia/codigos/exemplo_coffeescript_api_google-maps-for-rails_adaptado.js.coffee}

  \end{block}

\end{frame}
