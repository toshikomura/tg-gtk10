\begin{frame}
\frametitle{Trabalhos Futuros}

  \begin{block}{Trabalhos Futuros}

   \begin{itemize}

    \item Para a primeira gema de exemplo \emph{\textbf{gemtranslatetoenglish}}:

    \begin{itemize}

     \item Abordar mais recursos da linguagem \emph{\textbf{Ruby}}.

     \item Desenvolver mais funcionalidades para mostrar mais modelos de teste.

    \end{itemize}

    \item Para a segunda gema \emph{\textbf{Google-Maps-For-Rails adaptada}}:

    \begin{itemize}

     \item Incluir a escolha do tipo de veículo para percorrer o caminho.

     \item Incluir a possibilidade de inserir pontos intermediários entre a origem e o destino.

     \item Incluir a possibilidade de trocar as cores entre os tracejados entre dois pontos do caminho.
     
     \item Incluir a possibilidade de mostrar detalhes do percurso.

     \item Incluir a possibilidade de escolher o tipo de caminho a ser percorrido
     (mais curto, mais longo, menor tempo, maior tempo, etc).

    \end{itemize}


   \end{itemize}

  \end{block}

\end{frame}


\begin{frame}
 \frametitle{Conclusão}

  \begin{block}{Conclusão}

   Aprendemos:

   \begin{itemize}

    \item Conceitos sobre as bibliotecas do \emph{\textbf{Ruby}}.

    \item Como criar uma biblioteca do \emph{\textbf{Ruby}}.

    \item Como adaptar uma biblioteca do \emph{\textbf{Ruby}}.

    \item Como adicionar funcionalidades em uma biblioteca do \emph{\textbf{Ruby}} que mapeia a
    \emph{\textbf{API}} do \emph{\textbf{Google Maps}}.

    \item A utilizar a \emph{\textbf{API}} do \emph{\textbf{Google Maps}}.

    \item Conceitos sobre a linguagem \emph{\textbf{CoffeeScript}}.

   \end{itemize}

  \end{block}

\end{frame}
