\subsection{Conceito e Formas de Distribuição}
\begin{frame}
 \frametitle{Conceito e Formas de Distribuição}

  \begin{block}{Conceito}  
   Biblioteca do \emph{\textbf{Ruby}} é um conjunto de funções.
  \end{block}

 
  \begin{block}{Formas}

   \begin{itemize}

    \item Forma \emph{\textbf{arquivo compactado}} em .tar.gz ou .zip: Forma menos utilizada. Seu processo
    de instalação é feito por meio de arquivos README ou INSTALL.   
   
    \item Forma \emph{\textbf{gem}}: Forma mais utilizada para distribuição, por isso as bibliotecas do
    \emph{\textbf{Ruby}} são conhecidas como \emph{\textbf{gems/gemas}}. Seu processo de instalação
    é feito por meio do programa \emph{\textbf{gem}}.

   \end{itemize}

  \end{block}

  \begin{block}{Programa gem}

   \begin{itemize}

    \item Similar ao sistema de distribuição \emph{\textbf{apt-get}}.

    \item Executado em linhas de comando no formato ``\emph{\textbf{gem $<$operação$>$}}'',

    onde ``\emph{\textbf{$<$operação$>$}}'' pode ser \emph{\textbf{install}}, \emph{\textbf{search}}, etc.

   \end{itemize}

  \end{block}

\end{frame}


\subsection{Processo de Criação}
\begin{frame}
 \frametitle{Processo de Criação}

  \begin{block}{Processo}

    \begin{enumerate}

     \item Criar a estrutura automaticamente

      \lstinputlisting[ style=customCoffee, caption={Cria Gema Forma Geral}, label={lst:cria_gema_forma_geral}]
      {../monografia/codigos/cria_gema_forma_geral.sh}

     \item Editar ``\emph{\textbf{ `nome da gema'.gemspec}}'' para definir as descrições da gema.

     \item Desenvolver códigos de funcionalidades (diretório \emph{\textbf{lib}}, caso seja código \emph{\textbf{Ruby}}).

     \item Desenvolver códigos de testes (diretório \emph{\textbf{test}} ou \emph{\textbf{spec}}).
     
     \item Determinar versão para a \emph{\textbf{gema}}.
     
     \item Publicar a gema (\emph{\textbf{gem build ``nome da gema''}}).

    \end{enumerate}

  \end{block}

\end{frame}


\subsection{Uso da gema gemtranslatetoenglish}
\begin{frame}
 \frametitle{Uso da gema gemtranslatetoenglish}

  \begin{block}{Adiciona gema no Gemfile}

    Acrescenta a gema de tradução no Gemfile do projeto

    \lstinputlisting[ style=customCoffee, caption={Adiciona gemtranslatetoenglish no Gemfile}, label={lst:adiciona_gemtranslatetoenglish_no_gemfile}]
    {../monografia/codigos/adiciona_gemtranslatetoenglish_no_gemfile}

  \end{block}

  \begin{block}{Funcionalidade na View}

    Utiliza a funcionalidade de tradução da nova gema em uma view

    \lstinputlisting[ style=customCoffee, caption={Exemplo de gemtranslatetoenglish na View}, label={lst:exemplo_do_translate_na_view}]
    {../monografia/codigos/exemplo_do_translate_na_view.html.erb}

  \end{block}

\end{frame}
