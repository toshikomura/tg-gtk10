 \subsection{Objetivos do Trabalho}
 \begin{frame}
  \frametitle{Objetivos do Trabalho}

  \begin{block}{Objetivos}

   Apresentar:

   \begin{itemize}

    \item  Conceitos sobre as bibliotecas do \emph{\textbf{Ruby}}.

    \item  O processo de criação de uma biblioteca do \emph{\textbf{Ruby}}.

    \item  O processo de adaptação de uma biblioteca do \emph{\textbf{Ruby}}.

    \item  Exemplos de uso das bibliotecas do \emph{\textbf{Ruby}}.

    \begin{itemize}

     \item Uso da gema ``\emph{\textbf{gemtranslatetoenglish}}'': Tradução do português para o inglês.

     \item Uso da gema ``\emph{\textbf{Google-Maps-For-Rails Adaptada}}: Determina direções em uma
     biblioteca do \emph{\textbf{Ruby}} que mapeia a \emph{\textbf{API}} do
     \emph{\textbf{Google Maps}}.

    \end{itemize}

   \end{itemize}

  \end{block}

\end{frame}


 \subsection{Funcionalidades Google-Maps-For-Rails}
\begin{frame}
 \frametitle{Funcionalidades Google-Maps-For-Rails}

  \begin{block}{Funcionalidades}

   Por meio de código \emph{\textbf{CoffeeScript}}:

   \begin{itemize}

    \item  Cria mapas do \emph{\textbf{Google Maps}} em páginas web.

    \item  Permite a criação de mais de um mapa na mesma página (id).

    \item  Cria sobreposições para os mapas (markers, circles, polygon, polylines, etc).

    \item  Permite a criação de vários sobreposições do mesmo tipo com a chamada de uma única função.

    \item  Determina fronteiras no mapa, utilizando a posição de markers.

    \item  Permite readaptação do código \emph{\textbf{core}} para criar e manipular mapas de ferramentas
    similares ao do \emph{\textbf{Google}} (\emph{\textbf{Bing Maps}}),

   \end{itemize}

  \end{block}

\end{frame}


 \subsection{Proposta de Novas Funcionalidades}
\begin{frame}
 \frametitle{Proposta de Novas Funcionalidades}

  \begin{block}{Novas Funcionalidades}

   Por meio de código \emph{\textbf{CoffeeScript}}:

   \begin{itemize}

    \item  Gerar direções entre dois lugares no mapa.

    \item  Permitir a escolha do veículo de locomoção.

    \item  Gerar direções entre dois lugares com pontos intermediários obrigatórios.

    \item  Permitir a troca de cores entre os tracejados entre dois pontos do caminho.

    \item  Permitir a escolha do tipo de caminho a ser percorrido
    (mais curto, mais longo, menor tempo, maior tempo, etc).

   \end{itemize}

  \end{block}

\end{frame}
