 \subsection{Definição}
 \begin{frame}
  \frametitle{Definição}

  \begin{block}{Definição de Biblioteca}
    Conjunto de subprogramas ou rotinas que tem por função principal prover funcionalidades que geralmente são
    utilizadas por desenvolvedores em um determinado contexto.
  \end{block}

  \begin{block}{Vantagens do Uso de Bibliotecas}

    \begin{itemize}

     \item O uso de bibliotecas permite a reutilzação de código.

     \item O uso de bibliotecas permite a modularização do projeto.

    \end{itemize}

  \end{block}

\end{frame}


 \subsection{História}
\begin{frame}
 \frametitle{História}

  \begin{block}{História}

   \begin{itemize}

    \item \emph{\textbf{COMPOOL} (Communication Pool)}: \emph{Software} desenvolvido em \emph{\textbf{JOVIAL}}
   que tinha o objetivo de compartilhar informação entre vários programas.

    \item \emph{\textbf{COBOL}}: Linguagen de programção que em 1959 possuía um sistema primitivo de bibliotecas.

    \item \emph{\textbf{FORTRAN}}: Linguagen de programção que possuía um sistema que permitia a compilação de
   subprogramas independentemente dos outros.
   %ou seja, era permitido compilar o subprograma separadamente e depois
   %incluí-lo em outro programa. No entanto esse sistema enfraquecia o compilador que não consegui fazer a verificação
   %de tipos entre os subprogramas.

    \item \emph{\textbf{Simula 67}}: Linguagen de programção orientada a abjetos que em 1965 possibilatava o
   armazenamento de classes em arquivos de bibliotecas. Depois esses arquivos de bibliotecas poderiam ser
   incorporados em outros programas.

   \end{itemize}

  \end{block}

\end{frame}


 \subsection{Classificação}
\begin{frame}
 \frametitle{Classificação - Formas de Ligamento}

 \begin{block}{Definição}

  Processo que implica em associar uma biblioteca a um programa ou outra biblioteca.

 \end{block}

 \begin{block}{Tipos}

  \begin{itemize}

   \item \emph{\textbf{Tradicional}}: Os dados da biblioteca são copiados para dentro do executável do outro programa
   ou biblioteca.

   \item \emph{\textbf{Dinâmica}}: A biblioteca é referênciada no outro programa ou biblioteca. Neste caso não existe
   tanto trabalho no momento de compilação, pois a biblioteca só será copiada quando o programa for carregado ou outra
   biblioteca for carregada na memória.

   \item \emph{\textbf{Remoto}}: A biblioteca é carregada por chamadas de procedimentos remotos. A vantagem é que o
   programa e a biblioteca não precisam estar necessáriamente na mesma máquina, pois quando o programa requisitar
   a biblioteca, ela será transferida pela rede de uma máquina para outra.

  \end{itemize}

 \end{block}

\end{frame}


\begin{frame}
 \frametitle{Classificação - Momentos de Ligação}

 \begin{block}{Definição}

  É o momento em que a biblioteca é carregada na memória.

 \end{block}

 \begin{block}{Tipos}

  \begin{itemize}

   \item \emph{\textbf{Carregamento em Tempo de Carregamento}}: A biblioteca é carregada na memória no mesmo
   momento em que a aplicação esta sendo carregada.

   \item \emph{\textbf{Carregamento Dinâmica ou Atrasado}}: A biblioteca é carregada na memória somente quando
   a aplicação faz a requisição da biblioteca.

  \end{itemize}

 \end{block}

\end{frame}


\begin{frame}
 \frametitle{Classificação - Formas de Compartilhamento}

 \begin{block}{Definição}

  Disponiblidade da biblioteca para vários programas ao memso tempo.

 \end{block}

 \begin{block}{Tipos}

  \begin{itemize}

   \item \emph{\textbf{Compartilhamento em Disco}}: Programas podem utilizar o mesmo arquivo em disco para acessar a
   biblioteca.

   \begin{itemize}

     \item \emph{\textbf{Vantagens}}: Economia de tempo de compilação, espaço ocupado por binários e a possibilidade
     de usufruir de atualizações em a necessidade de recompilar o programa.

     %% Econmia de tempo de compilação e espaço, pois a biblioteca não é copiada para o o binário.

     \item \emph{\textbf{Desvantagens}}: Caso um biblioteca seja excluida ou danificada, todos os programas que a utilizam
     ela serão afetados.

   \end{itemize}

   \item \emph{\textbf{Compartilhamento em Memória}}: Programas podem compartilhar o acesso ao mesmo
   código da biblioteca na memória.

   \begin{itemize}

     \item \emph{\textbf{Vantagens}}: Economia de memória e tempo de carregamento.

      %% Ambiente multi-tarefa necessita de sincronização para garantir consistência dos dados.

     \item \emph{\textbf{Desvantagens}}: Desempenho pode ser prejudicado por causa da obrigatoriedade do ambiente
     multi-tarefa e a falta de escalabilidade.

      %% A falta de escalabilidade é por causa que a partir de uma certa quantidade de processadores não exste nenhum
   %% ganho de tempo

   \end{itemize}

  \end{itemize}

 \end{block}

\end{frame}
