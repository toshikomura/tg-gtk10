\subsection{Ferramentas Utilizadas}
\begin{frame}
 \frametitle{Ferramentas Utilizadas}

 \begin{block}{Ferramentas}
  \begin{itemize}

   \item \emph{\textbf{VMware® Player}}: aplicativo de virtualização que permite a utilização de vários SOs ao
  mesmo tempo sem a necessidade de reiniciar a máquina fisica.

   \item \emph{\textbf{RVM}}: ferramenta criada em outubro de 2007 por \emph{Wayne E. Seguin} que permite instalar,
  gerenciar, e trabalhar com multiplos ambiente do \emph{Ruby} com um certo conjunto de gemas.

   \item \emph{\textbf{Ruby On Rails}}: \emph{framework} criado em 2003 por \emph{David Heinemeier Hansson} e
  recebendo suporte do \emph{Rails Core Team}, facilita o desenvolvimento de código, visando a produtividade
  sustentável.

   \item \emph{\textbf{Git}}: ferramenta criada em 2005 por \emph{Linus Torvalds}, permite fazer o gerenciamento
  de versões tanto de projetos grandes como pequenos com rapidez e eficiência.

  \end{itemize}

 \end{block}

\end{frame}


\subsection{Preparação do ambiente}
\begin{frame}
 \frametitle{Preparação do ambiente}

  \begin{block}{Sequência de ferramentas a serem instaladas}

   \begin{itemize}

    \item Instalação do \emph{\textbf{VMware® Player}}.

    \item Instalação do \emph{\textbf{Ubuntu 12.04}} no \emph{\textbf{VMware® Playe}}.

    \item Instalação do \emph{\textbf{RVM}} no terminal dentro do \emph{\textbf{Ubuntu 12.04}}.

    \item Instalação do \emph{\textbf{Ruby}} também no terminal.

    \item Instalação das gema básicas \emph{\textbf{rails}} e \emph{\textbf{bundle}} também no terminal.

   \end{itemize}

  \end{block}

\end{frame}


\subsection{Criando uma gema}
%\subsubsection{Estrutura}
\begin{frame}
\frametitle{Estrutura Básica}

  \begin{block}{Estrutura de uma gema}

   \begin{itemize}

    \item \emph{\textbf{gemspec}}: arquivo que possui informações básicas da gema como nome, descrição,
    autor, endereço, e dependências.

    \item \emph{\textbf{bin}}: diretório que possui os executáveis que serão carregados quando a gema for
    instalada.

    \item \emph{\textbf{lib}}: diretório que possui todos os códigos \emph{\textbf{Ruby}} referentes ao
    funcionamento da gema.

    \item \emph{\textbf{spec}} ou \emph{\textbf{test}}: diretório que possui todos os códigos de teste da gema.

    \item \emph{\textbf{Rakefile}}: arquivo que possui código \emph{\textbf{Ruby}} que faz a otimização de
    algumas funcionalidades por meio da ferramenta \emph{\textbf{rake}}.

    \item \emph{\textbf{README}}: arquivo que possui a documentação da gema (\emph{\textbf{RDoc}} ou
    \emph{\textbf{YARD}} ).

   \end{itemize}

  \end{block}

\end{frame}


%\subsubsection{Criando a Estrutura}
\begin{frame}
\frametitle{Modelo de Criação - Criando a Estrutura}

  \begin{block}{Formas de criação}

   \begin{itemize}

    \item \emph{\textbf{Manual}}: Implica na criação de todos os arquivo e diretórios manuallmente.

    \item \emph{\textbf{Automática}}: Implica na execução do comando 
    ‘‘ \emph{\textbf{bundle gem ‘‘nome da gema''}} ''.

   \end{itemize}

  \end{block}

\end{frame}


%\subsubsection{Gemspec}
\begin{frame}
\frametitle{Modelo de Criação - Gemspec}

  \begin{block}{Conteúdo do arquivo gemspec}

   \begin{itemize}

    \item Nome da gema.

    \item Versão da gema.

    \item Autores da gema.

    \item E-mail dos autores e/ou e-mail para notificações sobre a gema.

    \item Breve descrição da gema.

    \item Descrição completa da gema.

    \item Localização dos arquivos e diretórios da estrtura básica da gema.

    \item Dependências da gema.

   \end{itemize}

  \end{block}

\end{frame}


\begin{frame}
\frametitle{Modelo de Criação -  Funcionalidades ou Testes}

  \begin{block}{Escolha de desenvolvimento}

   \begin{itemize}

    \item Depende da politica de desenvolvimento adotada no inicio do projeto.

    \item Desenvolver código de funcionalidade e depois desenvolver o código de teste.

    \item Desenvolver código de teste e depois desenvolver o código de funcionalidade.

   \end{itemize}

  \end{block}

\end{frame}


\begin{frame}
\frametitle{Modelo de Criação - Versão da gema}

  \begin{block}{Formas de Versionamento}

   \begin{itemize}

    \item \emph{\textbf{PATH}}: ‘‘\emph{0.0.X}'' para pequenas correções como correções de bugs.

    \item \emph{\textbf{MINOR}}: ‘‘\emph{0.X.0}'' para médias alterações como alterações de funcionalidades.

    \item \emph{\textbf{MAJOR}}: ‘‘\emph{X.0.0}'' para grandes modificações como remoção de funcionalidades.

    \item \emph{\textbf{PRE}}: ‘‘\emph{0.0.0-pre}'' para pré-lançamentos de versões.

   \end{itemize}

  \end{block}

\end{frame}


\begin{frame}
\frametitle{Modelo de Criação - Diferença entre module e class}

  \begin{block}{Module}

   \begin{itemize}

    \item Conjunto de métodos e constantes.

    \item Os métodos são estáticos.

    \item Sinilaridade com o conceito de \emph{\textbf{interface}} do \emph{\textbf{JAVA}}.

   \end{itemize}

  \end{block}

  \begin{block}{Class}

   \begin{itemize}

    \item Conjunto de métodos e constantes.

    \item Para utlizar os métodos e constantes é necessário instanciar um objeto na memória.

    \item \emph{\textbf{class}} é uma subclasse de \emph{\textbf{module}} (‘‘\emph{\textbf{initialize()}}'',
    ‘‘\emph{\textbf{superclass()}}'', ‘‘\emph{\textbf{allocate()}}'' e ‘‘\emph{\textbf{to\_yank()}}'').

   \end{itemize}

  \end{block}

\end{frame}


%\subsubsection{Código de Funcionalidade}
\begin{frame}
\frametitle{Modelo de Criação - Código de Funcionalidade}

  \begin{block}{Desenvolvimento de funcionalidades}

   \begin{itemize}

    \item As funcionalidades da gema desenvolvidas em código \emph{\textbf{Ruby}} devem ser inseridas no
    diretório \emph{\textbf{lib}}.

    \item As funcionalidades da gema desenvolvidas em código \emph{\textbf{Javascript}} devem ser inseridas no
    diretório \emph{\textbf{vendor}}.

    \item Arquivo (\emph{\textbf{lib/‘‘nome da gema''.rb}}).

    \item Modularizar a gema fazendo uso de \emph{\textbf{require}}.

    \item Diretório (\emph{\textbf{lib/‘‘nome da gema''/}}).

   \end{itemize}

  \end{block}

\end{frame}


%\subsubsection{Código de Teste}
\begin{frame}
\frametitle{Modelo de Criação - Código de Teste}

   \begin{block}{Desenvolvimento de testes}

   \begin{itemize}

    \item Os testes da gema desenvolvidas devem ser inseridos no diretório \emph{\textbf{test}} ou
    \emph{\textbf{spec}}.

    \item É recomendado nomear o arquivo de teste com ‘‘ \emph{\textbf{teste\_‘‘funcionalidade a ser testada''}} ''.

    \item Uso de \emph{\textbf{asserts}},

    \item Utilizar o arquivo ‘‘\emph{\textbf{Rakefile}}'' para automatizar os testes.

   \end{itemize}

  \end{block}

\end{frame}


%\subsubsection{Execução de Teste}
\begin{frame}
\frametitle{Modelo de Criação - Execução de Teste}

  \begin{block}{Realizar Testes}

   \begin{itemize}

   \item Fazer o ‘‘\emph{\textbf{build}} e a instalação da gema.

    \item Utilização da ferramenta \emph{\textbf{rake}}.

    \item Realizar testes por arquivo de teste com o comando ‘‘\emph{\textbf{rake test/nome do arquivo}}'' ou
    ‘‘\emph{\textbf{rake spec/nome do arquivo}}''.

    \item Realizar teste utilizando todos os arquivos de teste executando o comando ‘‘\emph{\textbf{rake}}''.

   \end{itemize}

  \end{block}

\end{frame}


\subsection{Adaptando uma gema}
\begin{frame}
\frametitle{Adaptando uma gema - Observações}

  \begin{block}{Observações}

   \begin{itemize}

    \item Adição de funcionalidades que geralmente são utilizadas.

    \item Funcionalidade a ser adicionada deve estar no mesmo contexto da gema.

     \begin{itemize}

      \item \textbf{CORRETO}: Adição do cálculo da raiz quadrada de um número em uma gema que faz cáculos.

      \item \textbf{ERRADO}: Adição da criação de mapas em uma gema que faz cálculos.

     \end{itemize}

    \item Evita implementações do zero.

   \end{itemize}

  \end{block}

\end{frame}


\begin{frame}
\frametitle{Adaptando uma gema - Engenharia Reversa}

  \begin{block}{Engenharia Reversa}

   \begin{itemize}

    \item É um processo de análise para a extração de informações de algo que já
    existe em um modelo de abstração de alto nível.

    \item As informações podem estar no formato de código fonte ou mesmo em um executável.

    \item O processo de análise para a extração de dados deve ser feita de forma
    minunciosa, pois pode ocorrer uma grande perda de recursos, caso alguma funcionalidade seja entendida de
    forma incorreta.

    \item O modelo de abstração de alto nível pode ser por exemplo um diagrama de caso de uso ou
    um diagrama de sequência.

   \end{itemize}

  \end{block}

\end{frame}


\begin{frame}
\frametitle{Adaptando uma gema - Google Maps e sua API}

  \begin{block}{Google Maps e sua API}

   \begin{itemize}

    \item Criada por dois irmãos \emph{\textbf{Lars}} e \emph{\textbf{Jens Rasmussen}}.

    \item A rederização de mapas até 2005 possuía um alto custo, nessitando de servidores altamente equipados.

    \item O \emph{\textbf{Google Maps}} foi anunciado em Fevereiro de 2005 no blog do \emph{Google}.

    \item Em Junho de 2005 o \emph{Google} anunciou a primeira \emph{\textbf{API}} do \emph{\textbf{Google Maps}},
    pois percebeu que os usuários gostariam de incorporar os mapas em seus sites.

    \item Para fazer modificações na gema de exemplo foi necessário consultar o livro
    \emph{\textbf{Beginning Google Maps API 3}} e o \emph{\textbf{Google Maps API V3}}.

   \end{itemize}

  \end{block}

\end{frame}


\begin{frame}
\frametitle{Adaptando uma gema - Entendimento e Adaptação}

  \begin{block}{Entendimento da gema}

   \begin{itemize}

    \item Verificar se realmente os diagramas estão de acordo com as características da gema.

    \item Encontrar os principais elementos da gema.

    \item Encontrar os elementos mais utilizados ou mais reaproveitadas da gema.

   \end{itemize}

  \end{block}

  \begin{block}{Adaptação da gema}

   \begin{itemize}

    \item Encontrar as funcionalidades que podem ser reaproveitadas na inclusão das novas funcionalidades.

    \item Após implementação das novas funcionalidades, verificar o impacto que elas causaram na gema.

   \end{itemize}

  \end{block}

\end{frame}
