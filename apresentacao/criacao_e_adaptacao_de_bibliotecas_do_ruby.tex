\subsection{Ferramentas Utilizadas}
\begin{frame}
 \frametitle{Ferramentas Utilizadas}

 \begin{itemize}
 
  \item \emph{\textbf{VMware® Player}}: aplicativo de virtualização que permite a utilização de vários SOs ao 
  mesmo tempo sem a necessidade de reiniciar a máquina fisica.
  
  \item \emph{\textbf{RVM}}: ferramenta criada em outubro de 2007 por \emph{Wayne E. Seguin} que permite instalar,
  gerenciar, e trabalhar com multiplos ambiente do \emph{Ruby} com um certo conjunto de gemas.	
  
  \item \emph{\textbf{Ruby On Rails}}: \emph{framework} criado em 2003 por \emph{David Heinemeier Hansson} e
  recebendo suporte do \emph{Rails Core Team}, facilita o desenvolvimento de código, visando a produtividade 
  sustentável.
  
  \item \emph{\textbf{Git}}: ferramenta criada em 2005 por \emph{Linus Torvalds}, permite fazer o gerenciamento 
  de versões tanto de projetos grandes como pequenos com rapidez e eficiência.
  
 \end{itemize}

 
\end{frame}

\subsection{Preparação do ambiente}
\begin{frame}
 \frametitle{Preparação do ambiente}
 Texto
\end{frame}

\subsection{Criando uma gema}
%\subsubsection{Estrutura}
\begin{frame}
\frametitle{Estrutura Básica}
 Texto
\end{frame}

%\subsubsection{Criando a Estrutura}
\begin{frame}
\frametitle{Modelo de Criação - Criando a Estrutura}
 Texto
\end{frame}

%\subsubsection{Gemspec}
\begin{frame}
\frametitle{Modelo de Criação - Gemspec}
 Texto
\end{frame}

%\subsubsection{Código de Funcionalidade}
\begin{frame}
\frametitle{Modelo de Criação - Código de Funcionalidade}
 Texto
\end{frame}

%\subsubsection{Código de Teste}
\begin{frame}
\frametitle{Modelo de Criação - Código de Teste}
 Texto
\end{frame}

%\subsubsection{Execução de Teste}
\begin{frame}
\frametitle{Modelo de Criação - Execução de Teste}
 Texto
\end{frame}

%\subsubsection{Exemplo de Uso}
\begin{frame}
\frametitle{Modelo de Criação - Exemplo de Uso}
 Texto
\end{frame}

\subsection{Adaptando uma gema}
\begin{frame}
\frametitle{Adaptando uma gema}
Texto
\end{frame}