\documentclass[10pt]{beamer}

\usepackage[portuguese]{babel}
\usepackage[utf8]{inputenc}
\usepackage{beamerthemesplit}
\usepackage{graphicx}

\usetheme{Warsaw}

\title{
  Criando e Adaptando \\ 
  Gemas do Ruby On Rails
}

\author{Gustavo Toshi Komura}
\institute{
  Universidade Federal do Paraná - UFPR \\
  Ciência da Computação \\
  Departamento de Informática \\
  Orientador: Professor Doutor Bruno Müller Junior  
}

\date{Curitiba - \today}



\begin{document}


\begin{frame}
 \titlepage
\end{frame}

\tableofcontents

 \section{Bibliotecas}
 \subsection{Definição}
 \begin{frame}
  \frametitle{Definição}
  
 \end{frame}

 
 \subsection{História}
\begin{frame}
 \frametitle{História}
 
  \begin{itemize}
  
   \item \emph{\textbf{COMPOOL} (Communication Pool)}: \emph{software} desenvolvido em \emph{\textbf{JOVIAL}} que tinha 
   o objetivo de compartilhar informação entre vários programas.
   
   \item \emph{\textbf{COBOL}}: linguagen de programção que em 1959 possuía um sistema primitivo de bibliotecas.
   
   \item \emph{\textbf{FORTRAN}}: linguagen de programção que possuía um sistema que permitia a compilação de 
   subprogramas independentemente dos outros. 
   %ou seja, era permitido compilar o subprograma separadamente e depois 
   %incluí-lo em outro programa. No entanto esse sistema enfraquecia o compilador que não consegui fazer a verificação 
   %de tipos entre os subprogramas.
   
   \item \emph{\textbf{Simula 67}}: linguagen de programção orientada a abjetos que em 1965 possibilatava o 
   armazenamento de classes em arquivos de bibliotecas. Depois esses arquivos de bibliotecas poderiam ser 
   incorporados em outros programas.
   
  \end{itemize}

  
\end{frame}

 \subsection{Classificação}
\begin{frame}
 \frametitle{Classificação - Formas de Ligamento}
 Texto
\end{frame}

\begin{frame}
 \frametitle{Classificação - Momentos de Ligação}
 Texto
\end{frame}

\begin{frame}
 \frametitle{Classificação - Formas de Compartilhamento}
 Texto
\end{frame}

\section{Ruby e suas bibliotecas}
 \subsection{Ruby - Definição} 
\begin{frame}
 \frametitle{Ruby - Definição}
 Texto
\end{frame}

\subsection{Ruby - História} 
\begin{frame}
 \frametitle{Ruby - História}
 Texto
\end{frame}


\subsection{Bibliotecas do Ruby} 
\begin{frame}
 \frametitle{Bibliotecas do Ruby}
 Texto
\end{frame}

\subsection{Segurança das Bibliotecas} 
\begin{frame}
 \frametitle{Segurança das Bibliotecas}
 Texto
\end{frame}

\section{Criado e Adaptando uma gema do Ruby}
 \subsection{Ferramentas Utilizadas}
\begin{frame}
 \frametitle{Ferramentas Utilizadas}

 \begin{itemize}
 
  \item \emph{\textbf{VMware® Player}}:
  
  \item \emph{\textbf{RVM}}:
  
  \item \emph{\textbf{Ruby On Rails}}:
  
  \item \emph{\textbf{Git}}:
  
 \end{itemize}

 
\end{frame}

\subsection{Preparação do ambiente}
\begin{frame}
 \frametitle{Preparação do ambiente}
 Texto
\end{frame}

\subsection{Criando uma gema}
%\subsubsection{Estrutura}
\begin{frame}
\frametitle{Estrutura Básica}
 Texto
\end{frame}

%\subsubsection{Criando a Estrutura}
\begin{frame}
\frametitle{Modelo de Criação - Criando a Estrutura}
 Texto
\end{frame}

%\subsubsection{Gemspec}
\begin{frame}
\frametitle{Modelo de Criação - Gemspec}
 Texto
\end{frame}

%\subsubsection{Código de Funcionalidade}
\begin{frame}
\frametitle{Modelo de Criação - Código de Funcionalidade}
 Texto
\end{frame}

%\subsubsection{Código de Teste}
\begin{frame}
\frametitle{Modelo de Criação - Código de Teste}
 Texto
\end{frame}

%\subsubsection{Execução de Teste}
\begin{frame}
\frametitle{Modelo de Criação - Execução de Teste}
 Texto
\end{frame}

%\subsubsection{Exemplo de Uso}
\begin{frame}
\frametitle{Modelo de Criação - Exemplo de Uso}
 Texto
\end{frame}

\subsection{Adaptando uma gema}
\begin{frame}
\frametitle{Adaptando uma gema}
Texto
\end{frame}

\section{Conclusão}
\begin{frame}
 \frametitle{Conclusão}
 Texto
\end{frame}

\end{document}