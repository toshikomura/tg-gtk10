\documentclass[10pt]{beamer}

\usepackage[portuguese]{babel}
\usepackage[utf8]{inputenc}
\usepackage{cmap}
\usepackage{beamerthemesplit}
\usepackage{color}
\usepackage{graphicx}
\usepackage{listings}
\lstset{%
  literate=
    {Á}{{\'{A}}}1
    {á}{{\'{a}}}1
}

\definecolor{mygreen}{RGB}{0,120,0}
\definecolor{myblue}{RGB}{0,0,163}
\renewcommand{\lstlistingname}{Código}

\lstdefinestyle{customCoffee}{
   extendedchars=true,
   language=Bash,
   emphstyle={\color{mygreen}},
   backgroundcolor=\color{white},
   basicstyle=\footnotesize\ttfamily,
   breakatwhitespace=false,
   breaklines=true,
   captionpos=b,
   commentstyle=\color{gray},
   keywordstyle=\color{blue},
   morecomment=[l]{//},
   morecomment=[s]{/*}{*/},
   morekeywords={and},
   extendedchars=true,
   frame=single,
   keepspaces=true,
   numbers=left,
   numberstyle=\tiny\color{black},
   rulecolor=\color{black},
   showspaces=false,
   showstringspaces=false,
   showtabs=false,
   stepnumber=1,
   stringstyle=\color{magenta},
   tabsize=1
}


\usetheme{Warsaw}

\title{
  Criando e Adaptando \\
  Bibliotecas do Ruby
}

\author{Gustavo Toshi Komura}
\institute{
  Universidade Federal do Paraná - UFPR \\
  Ciência da Computação \\
  Departamento de Informática \\
  Orientador: Professor Doutor Bruno Müller Junior
}

\date{Curitiba - \today}

\begin{document}


\begin{frame}
 \titlepage
\end{frame}

\begin{frame}
 \frametitle{Sumário}
 \tableofcontents
\end{frame}

 \section{Introdução}
 Até o ano de 2005, a rederização de mapas em uma página web custava caro. Isso ocorria,
porque para apresentar um mapa, era necessário fazer a transferência de uma grande
quantidade de dados do servidor para o cliente. Isso ocorria, porque era feito a
transferência de um mapa inteiro. Também como consquência da grande quantidade de dados
transferidos, existia um alto custo no cliente para transformar os dados em um mapa.

Graças ao \emph{Google}, em Fevereiro de 2005, essa dor de cabeça foi aliviada com a criação
de uma nova solução de rederização, implementada na ferramenta \emph{Google Maps}. Esta nova
solução, consistia na divisão do mapa em pedaços. Estes pedaços do mapa, poderiam ser
requisitados um a um para formar uma parte ou um mapa completo. Com essa nova rederização,
foi possível solucionar o problema da grande quantidade de dados transferidos, pois somente
se transfere as partes requisitadas do mapa, e também se solucionou o problema da transformação,
pois a quantidade de dados para se transformar, foi reduzida razoavelmente.

Muitas companias, percebendo a solução revolucionária adotada pelo \emph{Google} na
ferramenta \emph{Google Maps}, não ficaram atrás, adotando soluções sinilares que
também reduziam a carga de trabalho nos servidores. Entre estas ferramentas estão, o
\emph{Yahoo Maps} do \emph{Yahoo}, o \emph{Bing Maps} da \emph{Microsoft}, e o
\emph{Yandex Maps} do \emph{Yandex}.

Para os desenvolvedores, esta nova solução de rederização, tornou-se interessante
quando se analisou as operações que poderiam ser realizadas sobre os mapas. Por exemplo, nos
mapas do \emph{Google Maps}, podemos criar um mapa em uma página web, utilizar marcadores para
marcar locais ou regiões do mapa que consideramos importante, e até determinar caminhos
para ir de um local ao outro.

Muitos programadores decidiram se mobilizar na procura de uma forma de incorporar o
\emph{Google Maps} em suas aplicações, pois perceberam que a ferramenta possuía funcionalidade
muito úteis para o dia-a-dia. O \emph{Google}, percebendo essa mobilização, resolveu facilitar
o acesso a ferramenta, divulgando em Junho de 2005 uma \emph{API}
(\emph{Application Programming Interface}) para o \emph{Google Maps}, apresentando
detalhes de como importar e utilizar as funções da ferramenta.

O \emph{Ruby} possui bibliotecas que contém funcionalidade prontas para uso. Algumas destas
bibliotecas possuem funcionalidades que simplificam o acesso da \emph{API} do \emph{Google Maps}.
Basicamente estas biblioteca mapeiam as funções da ferramenta do \emph{Google}, preprando
internamente os objetos dos mapas. Deste modo, ao utilizar uma funcionalidade da biblioteca, não
é necessários saber quais os objetos do \emph{Google Maps} precisam ser criados, pois estes
objetos serão gerados automaticamente na chamada da função.

O \emph{Ruby On Rails} é um \emph{framework} da linguagem \emph{Ruby} que facilita o
desenvolvimento de projetos \emph{web}, fazendo a criação de estrturas básicas por meio da
execução de linhas de comando, e possibilitando a importação de bibliotecas do \emph{Ruby}.

O objetivo deste trabalho, é mostrar uma forma de adaptar uma biblioteca do \emph{Ruby}
que faz o mapeamento da \emph{API} do \emph{Google Maps} e apresentar um exemplo de uso desta
biblioteca em um projeto do \emph{Ruby On Rails}. Para aprimorar o conhecimento, antes será
necessário apresentar conceitos sobre bibliotecas do \emph{Ruby}, bem como o seu modelo de
criação.

O texto apresentará um passo-a-paaso da implementação de uma biblioteca simples
e funcional do \emph{Ruby}. Como exemplo, será desenvolvido um biblioteca que faz a
tradução de palavras do português para o inglês. E utilizando este conhecimento de criação
de bibliotecas, o texto também apresentará os passos básicos que devem ser realizados
para adaptar uma biblioteca do \emph{Ruby}. Como exemplo, será feito a adição da
funcionalidade de desenhar caminhos entre dois locais, utilizando uma biblioteca do
\emph{Ruby} que mapeia a \emph{API} do \emph{Google Maps}. 

Este trabalho está organizado da seguinte maneira: bibliotecas do \emph{Ruby} no
capítulo \ref{chapter:bibliotecas_do_ruby}, onde será apresentado conceitos sobre
bibliotecas do \emph{Ruby}, criação de bibliotecas do \emph{Ruby} no capítulo 
\ref{chapter:criacao_de_bibliotecas_do_ruby}, onde será apresentado um passo-a-passo
de como criar uma biblioteca do \emph{Ruby}, adaptação de bibliotecas do \emph{Ruby}
no capítulo  \ref{chapter:adaptacao_de_bibliotecas_do_ruby}, onde será apresentado um
tutorial de como adaptar uma biblioteca do \emph{Ruby}, e a conclusão no capítulo
\ref{chapter:conclusao}.

\section{Bibliotecas do Ruby}
 \subsection{Conceito e Formas de Distribuição}
\begin{frame}
 \frametitle{Conceito e Formas de Distribuição}

  \begin{block}{Conceito}  
   Biblioteca do \emph{\textbf{Ruby}} é um conjunto de funções.
  \end{block}

 
  \begin{block}{Formas}

   \begin{itemize}

    \item Forma \emph{\textbf{arquivo compactado}} em .tar.gz ou .zip: Forma menos utilizada. Seu processo
    de instalação é feito por meio de arquivos README ou INSTALL.   
   
    \item Forma \emph{\textbf{gem}}: Forma mais utilizada para distribuição, por isso as bibliotecas do
    \emph{\textbf{Ruby}} são conhecidas como \emph{\textbf{gems/gemas}}. Seu processo de instalação
    é feito por meio do programa \emph{\textbf{gem}}.

   \end{itemize}

  \end{block}

  \begin{block}{Programa gem}

   \begin{itemize}

    \item Similar ao sistema de distribuição \emph{\textbf{apt-get}}.

    \item Executado em linhas de comando no formato ``\emph{\textbf{gem $<$operação$>$}}'',

    onde ``\emph{\textbf{$<$operação$>$}}'' pode ser \emph{\textbf{install}}, \emph{\textbf{search}}, etc.

   \end{itemize}

  \end{block}

\end{frame}


\subsection{Processo de Criação}
\begin{frame}
 \frametitle{Processo de Criação}

  \begin{block}{Processo}

    \begin{enumerate}

     \item Criar a estrutura automaticamente

      \lstinputlisting[ style=customCoffee, caption={Cria Gema Forma Geral}, label={lst:cria_gema_forma_geral}]
      {../monografia/codigos/cria_gema_forma_geral.sh}

     \item Editar ``\emph{\textbf{ `nome da gema'.gemspec}}'' para definir as descrições da gema.

     \item Desenvolver códigos de funcionalidades (diretório \emph{\textbf{lib}}, caso seja código \emph{\textbf{Ruby}}).

     \item Desenvolver códigos de testes (diretório \emph{\textbf{test}} ou \emph{\textbf{spec}}).
     
     \item Determinar versão para a \emph{\textbf{gema}}.
     
     \item Publicar a gema (\emph{\textbf{gem build ``nome da gema''}}).

    \end{enumerate}

  \end{block}

\end{frame}


\subsection{Uso da gema gemtranslatetoenglish}
\begin{frame}
 \frametitle{Uso da gema gemtranslatetoenglish}

  \begin{block}{Adiciona gema no Gemfile}

    Acrescenta a gema de tradução no Gemfile do projeto

    \lstinputlisting[ style=customCoffee, caption={Adiciona gemtranslatetoenglish no Gemfile}, label={lst:adiciona_gemtranslatetoenglish_no_gemfile}]
    {../monografia/codigos/adiciona_gemtranslatetoenglish_no_gemfile}

  \end{block}

  \begin{block}{Funcionalidade na View}

    Utiliza a funcionalidade de tradução da nova gema em uma view

    \lstinputlisting[ style=customCoffee, caption={Exemplo de gemtranslatetoenglish na View}, label={lst:exemplo_do_translate_na_view}]
    {../monografia/codigos/exemplo_do_translate_na_view.html.erb}

  \end{block}

\end{frame}


\section{Bibliotecas do Ruby Mapeando API Google Maps}
 \subsection{API Google-Maps-For-Rails}
\begin{frame}
 \frametitle{API Google-Maps-For-Rails}

      \lstinputlisting[ style=customCoffee, caption={Exemplo CoffeeScript API Google-Maps-For-Rails}, label={lst:exemplo_coffeescript_api_google-maps-for-rails}]
      {../monografia/codigos/exemplo_coffeescript_api_google-maps-for-rails.js.coffee}

\end{frame}


\subsection{Processo de Adaptação}
\begin{frame}
 \frametitle{Processo de Adaptação}

  \begin{block}{Processo}

    \begin{enumerate}

     \item Realizar \emph{\textbf{engenharia reversa}} para gerar diagramas de alto nível.

     \item Entender a biblioteca por meio dos diagramas de alto nível.

     \item Realizar adaptações.

    \end{enumerate}

  \end{block}

\end{frame}


\subsection{API Google-Maps-For-Rails Adaptada}
\begin{frame}
 \frametitle{API Google-Maps-For-Rails Adaptada}

  \begin{block}{Criar Direções}

   Por meio de código \emph{\textbf{CoffeeScript}}:

      \lstinputlisting[ style=customCoffee, caption={Exemplo CoffeeScript API Google-Maps-For-Rails Adaptado}, label={lst:exemplo_coffeescript_api_google-maps-for-rails_adaptado}]
      {../monografia/codigos/exemplo_coffeescript_api_google-maps-for-rails_adaptado.js.coffee}

  \end{block}

\end{frame}


\section{Conclusão}
 Neste trabalho apresentamos alguns conceitos básicos sobre bibliotecas, descrevendo a imprtância de seu
uso no capitulo \ref{chap:historico_e_concetios_de_bibliotecas}. Depois dedicamos um pouco do tempo
no capítulo \ref{chap:ruby_e_suas_bibliotecas} para falar sobre a linguagem \emph{Ruby}, suas
bibliotecas, e a possibilidade de se utilzar chaves para se ter segurança na instalação destas bibliotecas.

Percebendo que somente a utilização de bibliotecas de terceiros, as vezes não é o suficiente para se
ter economia de tempo, passamos para um outro nível, aonde possibilitamos uma equipe ou mesmo somente
uma pessoa a desenvolver a sua própia biblioteca quando uma certa funcionalidade é muito utlizada em vários
tipos de projetos.

Contudo no capítulo \ref{chap:criacao_e_adaptacao_de_bibliotecas_do_ruby} apresentamos um tutorial
básico de como se pode criar ou modificar uma biblioteca do \emph{Ruby}, descrevendo em detalhes os
passos de implementação que devem ser seguidos para se ter mais chances de conseguir sucesso no
projeto. Neste capítulos também detalhamos as ferramentas e os comandos utilizados durante o processo de
desenvolvimento apresentando exemplos para facilitar a compreensão do tutorial.

Acreditamos que com a apresentação deste trabalho, uma equipe dependendo das suas necessidades tem
a possibilidade de desenvolver do zero ou adaptar bibliotecas do \emph{Ruby}, podendo assim
economizar tempo de desenvolvimento em projetos futuros.

Para este trabalho no exemplo de criação de \emph{gemas}, desenvolvemos uma \emph{gema} simples de
tradução, mostrando somente alguns detalhes da linguagem \emph{Ruby}, e por esse motivo para
trabalhos futuros, poderia ser criada uma nova biblioteca ou mesmo fazer uma adaptação da
\emph{gema} de exemplo para mostrar mais conceitos da linguagem. Também para a biblioteca
\emph{Google-Maps-For-Rails} adaptada para aceitar a funcionalidade de gerar direções, poderia ser feito o
incremento de mais funcionalidades, como por exemplo, a inclusão da escolha do tipo de locomoção a ser
utilizada no percurso e a possibilidade de adicionar locais intermediários no caminho.


\end{document}
