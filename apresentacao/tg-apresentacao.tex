\documentclass[10pt]{beamer}

\usepackage[portuguese]{babel}
\usepackage[utf8]{inputenc}
\usepackage{cmap}
\usepackage{beamerthemesplit}
\usepackage{color}
\usepackage{graphicx}
\usepackage{listings}
\lstset{%
  literate=
    {Á}{{\'{A}}}1
    {á}{{\'{a}}}1
}

\definecolor{mygreen}{RGB}{0,120,0}
\definecolor{myblue}{RGB}{0,0,163}
\renewcommand{\lstlistingname}{Código}

\lstdefinestyle{customCoffee}{
   extendedchars=true,
   language=Bash,
   emphstyle={\color{mygreen}},
   backgroundcolor=\color{white},
   basicstyle=\footnotesize\ttfamily,
   breakatwhitespace=false,
   breaklines=true,
   captionpos=b,
   commentstyle=\color{gray},
   keywordstyle=\color{blue},
   morecomment=[l]{//},
   morecomment=[s]{/*}{*/},
   morekeywords={and},
   extendedchars=true,
   frame=single,
   keepspaces=true,
   numbers=left,
   numberstyle=\tiny\color{black},
   rulecolor=\color{black},
   showspaces=false,
   showstringspaces=false,
   showtabs=false,
   stepnumber=1,
   stringstyle=\color{magenta},
   tabsize=1
}


\usetheme{Warsaw}

\title{
  Criando e Adaptando \\
  Bibliotecas do Ruby
}

\author{Gustavo Toshi Komura}
\institute{
  Universidade Federal do Paraná - UFPR \\
  Ciência da Computação \\
  Departamento de Informática \\
  Orientador: Professor Doutor Bruno Müller Junior
}

\date{Curitiba - \today}

\begin{document}


\begin{frame}
 \titlepage
\end{frame}

\begin{frame}
 \frametitle{Sumário}
 \tableofcontents
\end{frame}

 \section{Introdução}
  \subsection{Objetivos do Trabalho}
 \begin{frame}
  \frametitle{Objetivos do Trabalho}

  \begin{block}{Objetivos}

   Apresentar:

   \begin{itemize}

    \item  Conceitos sobre as bibliotecas do \emph{\textbf{Ruby}}.

    \item  O processo de criação de uma biblioteca do \emph{\textbf{Ruby}}.

    \item  O processo de adaptação de uma biblioteca do \emph{\textbf{Ruby}}.
    
    \item  A adaptação de um biblioteca do \emph{\textbf{Ruby}} que faz o mapeamento
    da \emph{\textbf{API}} do \emph{\textbf{Google Maps}}.

    \item  Exemplos de uso das bibliotecas do \emph{\textbf{Ruby}}.

   \end{itemize}

  \end{block}

\end{frame}

\section{Bibliotecas do Ruby}
 \subsection{Conceitos}
\begin{frame}
 \frametitle{Conceitos}

  \begin{block}{Formas}

   \begin{itemize}

    \item Forma \emph{\textbf{arquivo compactado}} em .tar.gz ou .zip: Forma menos utilizada. Seu processo
    de instalação é feito por meio de arquivos README ou INSTALL.   
   
    \item Forma \emph{\textbf{gem}}: Forma mais utilizada para distribuição, por isso as bibliotecas do
    \emph{\textbf{Ruby}} são conhecidas como \emph{\textbf{gems/gemas}}. Seu processo de instalação
    é feito por meio do programa \emph{\textbf{gem}}.

   \end{itemize}

  \end{block}

  \begin{block}{Programa gem}

   \begin{itemize}

    \item Similar ao sistema de distribuição \emph{\textbf{apt-get}}.

    \item Executado em linhas de comando no formato ``\emph{\textbf{gem $<$operação$>$}}'',

    onde ``\emph{\textbf{$<$operação$>$}}'' pode ser:

    \begin{itemize}

     \item \emph{\textbf{install}} para fazer instalação de gemas.

     \item \emph{\textbf{search}} para fazer busca de gemas.

    \end{itemize}


   \end{itemize}

  \end{block}

\end{frame}


\begin{frame}
 \frametitle{Segurança}

  \begin{block}{Motivos}

   \begin{enumerate}

    \item Abrir brechas, correndo riscos de perder informações.

   \end{enumerate}

  \end{block}

  \begin{block}{Processo de Segurança do RubyGems}

   O \emph{\textbf{RubyGems}} é o repositório de distribuição padrão utilizado no
   programa \emph{\textbf{gem}}.

   \begin{enumerate}

    \item Certificação da gema.

    \item Verificação de segurança na instalação da gema.

   \end{enumerate}

  \end{block}

  \begin{block}{Reportar Vulnerabilidades}

   \begin{itemize}

    \item Vulnerabilidades nas gemas de outros desenvolvedores

    \item Vulnerabilidades das própias gemas
    
    \begin{itemize}
     \item Criar Identificador \emph{\textbf{CVE}}, criar patch e informar a comunidade.
     \item Registrar vulnerabilidade no banco \emph{\textbf{OSVBD}} e no \emph{\textbf{ruby-talk}}.
    \end{itemize}


   \end{itemize}

  \end{block}

\end{frame}


\subsection{Processo de Criação}
\begin{frame}
 \frametitle{Processo de Criação}

  \begin{block}{Processo}

    \begin{enumerate}

     \item Criar a estrutura automaticamente

      \lstinputlisting[ style=customCoffee, caption={Cria Gema Forma Geral}, label={lst:cria_gema_forma_geral}]
      {../monografia/codigos/cria_gema_forma_geral.sh}

     \item Editar ``\emph{\textbf{ `nome da gema'.gemspec}}'' para definir as descrições da gema.

     \item Desenvolver códigos de funcionalidades.

     \item Desenvolver códigos de testes.

    \end{enumerate}

  \end{block}

\end{frame}


\subsection{Uso da gema gemtranslatetoenglish}
\begin{frame}
 \frametitle{Uso da gema gemtranslatetoenglish}

  \begin{block}{Adiciona gema no Gemfile}

    Acrescenta a gema de tradução no Gemfile do projeto

    \lstinputlisting[ style=customCoffee, caption={Adiciona gemtranslatetoenglish no Gemfile}, label={lst:adiciona_gemtranslatetoenglish_no_gemfile}]
    {../monografia/codigos/adiciona_gemtranslatetoenglish_no_gemfile}

  \end{block}

  \begin{block}{Funcionalidade na View}

    Utiliza a funcionalidade de tradução da nova gema em uma view

    \lstinputlisting[ style=customCoffee, caption={Exemplo de gemtranslatetoenglish na View}, label={lst:exemplo_do_translate_na_view}]
    {../monografia/codigos/exemplo_do_translate_na_view.html.erb}

  \end{block}

\end{frame}


\section{Bibliotecas do Ruby Mapeando API Google Maps}
 \subsection{Processo de Adaptação}
\begin{frame}
 \frametitle{Processo de Adaptação}

  \begin{block}{Processo}

    \begin{enumerate}

     \item Realizar \emph{\textbf{engenharia reversa}} para gerar diagramas de alto nível.

     \item Entender a biblioteca por meio dos diagramas de alto nível.

     \item Realizar adaptações.

    \end{enumerate}

  \end{block}
  
\end{frame}


\subsection{Funcionalidades Google-Maps-For-Rails}
\begin{frame}
 \frametitle{Funcionalidades Google-Maps-For-Rails}
  
  \begin{block}{API Google Maps e Gema Google-Maps-For-Rails}       
  
    \begin{itemize}        
    
     \item \emph{\textbf{Google-Maps-For-Rails}} mapeia \emph{\textbf{API}} do \emph{\textbf{Google Maps}}.
     
     \item \emph{\textbf{API Google Maps}} utliza a linguagem \emph{\textbf{Javascript}}.     

     \item \emph{\textbf{Google-Maps-For-Rails}} utiliza \emph{\textbf{CoffeeScript}} -$>$ \emph{\textbf{Javascript}}.     

    \end{itemize}

  \end{block}    
 
  \begin{block}{Funcionalidades}   

   \begin{itemize}

    \item  Cria mapas do \emph{\textbf{Google Maps}} em páginas web.

    \item  Permite a criação de mais de um mapa na mesma página (id).

    \item  Cria sobreposições para os mapas (markers, circles, polygon, polylines, etc).

    \item  Permite a criação de vários sobreposições do mesmo tipo com a chamada de uma única função.    

    \item  Permite readaptação do \emph{\textbf{core}} (\emph{\textbf{Bing Maps}}).

   \end{itemize}

  \end{block}

\end{frame}


\subsection{API Google-Maps-For-Rails}
\begin{frame}
 \frametitle{API Google-Maps-For-Rails}

      \lstinputlisting[ style=customCoffee, caption={Exemplo CoffeeScript API Google-Maps-For-Rails}, label={lst:exemplo_coffeescript_api_google-maps-for-rails}]
      {../monografia/codigos/exemplo_coffeescript_api_google-maps-for-rails_apresentacao.js.coffee}

\end{frame}


 \subsection{Proposta de Novas Funcionalidades}
\begin{frame}
 \frametitle{Proposta de Novas Funcionalidades}

  \begin{block}{Novas Funcionalidades}   

   \begin{itemize}

    \item  Gerar direções entre dois lugares no mapa.    

    \item  Gerar direções entre dois lugares com pontos intermediários obrigatórios.

    \item  Permitir a troca de cores entre os tracejados entre dois pontos do caminho.
    
    \item  Permitir a escolha do veículo de locomoção.    
    
    \item  Mostrar detalhes sobre o caminho (tempo de percurso, distância, etc).

   \end{itemize}

  \end{block}

\end{frame}


\subsection{API Google-Maps-For-Rails Adaptada}
\begin{frame}
 \frametitle{API Google-Maps-For-Rails Adaptada}

  \begin{block}{Criar Direções}

   Por meio de código \emph{\textbf{CoffeeScript}}:

      \lstinputlisting[ style=customCoffee, caption={Exemplo CoffeeScript API Google-Maps-For-Rails Adaptado}, label={lst:exemplo_coffeescript_api_google-maps-for-rails_adaptado}]
      {../monografia/codigos/exemplo_coffeescript_api_google-maps-for-rails_adaptado.js.coffee}

  \end{block}

\end{frame}


\section{Conclusão}
 \begin{frame}
\frametitle{Trabalhos Futuros}
 
  \begin{block}{Trabalhos Futuros}
   
   \begin{itemize}
   
    \item Para a primeira gema de exemplo \emph{\textbf{gemtranslatetoenglish}}:
    
    \begin{itemize}
    
     \item Abordar mais recursos da linguagem \emph{\textbf{Ruby}}.
     
     \item Desenvolver mais funcionalidades para mostrar mais modelos de teste.
     
    \end{itemize}

    \item Para a segunda gema de exemplo \emph{\textbf{Google-Maps-For-Rails adaptada}}:
    
    \begin{itemize}
    
     \item Incluir a escolha de locomoção.
     
     \item Incluir a possibilidade de inserir pontos intermediários entre a origem e o destino.
     
    \end{itemize}

    
   \end{itemize}
   
  \end{block}
 
\end{frame}


\begin{frame}
 \frametitle{Conclusão}

  \begin{block}{Conclusão}
   
   \begin{itemize}
   
    \item Proporcionou o entendimento sobre bibliotecas.
    
    \item Mostrou conceitos básicos da linguagem \emph{\textbf{Ruby}}.
    
    \item Apresentou como criar uma biblioteca do \emph{\textbf{Ruby}}.
    
    \item Mostrou como adaptar uma biblioteca do \emph{\textbf{Ruby}}.
    
    \item Apresentou exemplos para facilitar o entendimento da explicação de como criar ou adaptar uma
    biblioteca do \emph{\textbf{Ruby}}.
    
    \item Proporcionou a possibilidade de economizar tempo no desenvolvimento.
    
   \end{itemize}      
   
  \end{block}
  
\end{frame}  

\end{document}
